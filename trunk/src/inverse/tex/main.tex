\documentclass[11pt,twoside]{article}
\usepackage{noweb,fullpage,clrscode,proof,hyperref,amsmath}
\usepackage{clf}

\title{Towards a Focusing Inverse-Method Theorem Prover for Canonical LF}
\author{Sean McLaughlin}

\begin{document} 
\maketitle

\begin{abstract} 
This is an experiment at implementing \emph{Canonical LF}.  We intend
to present muliple forms of representation.  We will begin with a
simple name-carrying implementation of Canonical LF.  We will continue
with spine forms and explicit substitutions, and finally add
meta-variables.  Through this process, we hope to experiment with the
various representations to find a good trade-off between efficiency
and elegance.
\end{abstract} 


%---------------------------------  Constants  ---------------------------------

\providecommand{\Nil}{\mathtt{nil}}
\providecommand{\Kind}{\mathtt{kind}}
\providecommand{\Type}{\mathtt{type}}
\providecommand{\IdSub}{\mathtt{id}}
\providecommand{\Of}[2]{#1\ :\ #2}

%-----------------------------------  Terms  -----------------------------------

\providecommand{\PiTyp}[3]{\Pi #1 : #2.\ #3}
\providecommand{\Lam}[2]{\lambda #1.\ #2}
\providecommand{\Appl}[2]{#1\ #2}
\providecommand{\Arr}[2]{#1\to #2}
\providecommand{\HSub}[4]{[#1/#2]^{#3}#4}


%--------------------------------  Separators  ---------------------------------

\providecommand{\Spb}{\ |\ }
\providecommand{\Shift}{\uparrow}
\providecommand{\Set}[1]{\{#1\}}
\providecommand{\Comp}{~\circ~}
\providecommand{\App}{~@~}
\providecommand{\SeqArr}{\Longrightarrow}
\providecommand{\LFArr}{\to}

%--------------------------------  Basic math  ---------------------------------

\providecommand{\set}[1]{\{#1\}}
\newcommand{\card}[1]{\left|#1\right|}
\providecommand{\Not}{\neg}
\providecommand{\And}{\wedge}
\providecommand{\Or}{\vee}
\providecommand{\AND}{\bigwedge}
\providecommand{\OR}{\bigvee}
\providecommand{\Imp}{\supset}
\providecommand{\Iff}{\Longleftrightarrow}
\providecommand{\Arr}{\Rightarrow}


\def\incomplete#1{XXX #1 XXX}





\input{../NW_TEX/spinelf.tex}
\input{../NW_TEX/translate.tex}

\appendix

\section{Structures}
%-------------------------------------------------------------------------------
% Contexts                                                                      
%-------------------------------------------------------------------------------

\subsection{Contexts}

Contexts map (DeBruijn) variables to types.  

\newcommand{\Ctx}{\ \mathtt{ctx}}
\bigskip 
\framebox{$\Gamma\Ctx$}
\bigskip 

$$
\begin{array}{cc}
\infer{\cdot\Ctx}{} &
\infer{\Gamma,A\Ctx}{\Gamma\Ctx & \CheckTy[\cdot]{A}{\Type}}
\end{array} 
$$


%-------------------------------------------------------------------------------
% Signatures                                                                    
%-------------------------------------------------------------------------------

\subsection{Signatures}

Signatures map constants to types and kinds.

\newcommand{\Sig}{\ \mathtt{sig}}
\bigskip 
\framebox{$\Sigma\Sig$}
\bigskip 

$$
\begin{array}{ccc}
\infer{\cdot\Sig}{}&
\infer{\Sigma,c:A\Sig}{\Sigma\Sig & \CheckTy[\cdot]{A}{\Type}}&
\infer{\Sigma,a:K\Sig}{\Sigma\Sig & \CheckTy[\cdot]{K}{\Kind}}
\end{array} 
$$


\subsection{Twelf Grammar}

see \texttt{twelf/src/lambda/intsyn.sig}

\begin{verbatim} 

  type cid = int			(* Constant identifier        *)
  type mid = int                        (* Structure identifier       *)
  type csid = int                       (* CS module identifier       *)

  datatype 'a Ctx =			(* Contexts                   *)
    Null				(* G ::= .                    *)
  | Decl of 'a Ctx * 'a			(*     | G, D                 *)

  datatype Depend =                     (* Dependency information     *)
    No                                  (* P ::= No                   *)
  | Maybe                               (*     | Maybe                *)
  | Meta				(*     | Meta                 *)

  datatype Uni =			(* Universes:                 *)
    Kind				(* L ::= Kind                 *)
  | Type				(*     | Type                 *)

  datatype Exp =			(* Expressions:               *)
    Uni   of Uni			(* U ::= L                    *)
  | Pi    of (Dec * Depend) * Exp	(*     | Pi (D, P). V         *)
  | Root  of Head * Spine		(*     | H @ S                *)
  | Redex of Exp * Spine		(*     | U @ S                *)
  | Lam   of Dec * Exp			(*     | lam D. U             *)
  | EVar  of Exp option ref * Dec Ctx * Exp * (Cnstr ref) list ref
                                        (*     | X<I> : G|-V, Cnstr   *)
  | EClo  of Exp * Sub			(*     | U[s]                 *)
  | AVar  of Exp option ref             (*     | A<I>                 *)

  | FgnExp of csid * FgnExp             (*     | (foreign expression) *)

  | NVar  of int			(*     | n (linear, 
                                               fully applied variable
                                               used in indexing       *)

  and Head =				(* Head:                      *)
    BVar  of int			(* H ::= k                    *)
  | Const of cid			(*     | c                    *)
  | Proj  of Block * int		(*     | #k(b)                *)
  | Skonst of cid			(*     | c#                   *)
  | Def   of cid			(*     | d (strict)           *)
  | NSDef of cid			(*     | d (non strict)       *)
  | FVar  of string * Exp * Sub		(*     | F[s]                 *)
  | FgnConst of csid * ConDec           (*     | (foreign constant)   *)

  and Spine =				(* Spines:                    *)
    Nil					(* S ::= Nil                  *)
  | App   of Exp * Spine		(*     | U ; S                *)
  | SClo  of Spine * Sub		(*     | S[s]                 *)

  and Sub =				(* Explicit substitutions:    *)
    Shift of int			(* s ::= ^n                   *)
  | Dot   of Front * Sub		(*     | Ft.s                 *)

  and Front =				(* Fronts:                    *)
    Idx of int				(* Ft ::= k                   *)
  | Exp of Exp				(*     | U                    *)
  | Axp of Exp				(*     | U                    *)
  | Block of Block			(*     | _x                   *)
  | Undef				(*     | _                    *)

  and Dec =				(* Declarations:              *)
    Dec of string option * Exp		(* D ::= x:V                  *)
  | BDec of string option * (cid * Sub)	(*     | v:l[s]               *)
  | ADec of string option * int	        (*     | v[^-d]               *)
  | NDec  

  and Block =				(* Blocks:                    *)
    Bidx of int				(* b ::= v                    *)
  | LVar of Block option ref * Sub * (cid * Sub)
                                        (*     | L(l[^k],t)           *)
  | Inst of Exp list                    (*     | U1, ..., Un          *)

  and ConDec =			        (* Constant declaration       *)
    ConDec of string * mid option * int * Status
                                        (* a : K : kind  or           *)
              * Exp * Uni	        (* c : A : type               *)
  | ConDef of string * mid option * int	(* a = A : K : kind  or       *)
              * Exp * Exp * Uni		(* d = M : A : type           *)
              * Ancestor                (* Ancestor info for d or a   *)
  | AbbrevDef of string * mid option * int
                                        (* a = A : K : kind  or       *)
              * Exp * Exp * Uni		(* d = M : A : type           *)
  | BlockDec of string * mid option     (* %block l : SOME G1 PI G2   *)
              * Dec Ctx * Dec list
  | SkoDec of string * mid option * int	(* sa: K : kind  or           *)
              * Exp * Uni	        (* sc: A : type               *)

\end{verbatim} 

\section{Using Twelf and User Code from the SML Top Level Loop}

This section describes using Twelf and user extensions from the SML/NJ top level
loop.


\subsection{Loading Files}
First, start sml:


\begin{verbatim} 
~/save/projects/twelf/my-twelf
\$ sml
Standard ML of New Jersey v110.59 [built: Wed Sep 20 23:04:52 2006]
- 
\end{verbatim} 

Next, cd to the twelf directory, and load Twelf via
the compilation manager using the command [[CM.make]].

\begin{verbatim} 
- CM.make "sources.cm"; 
[autoloading]
[library \$smlnj/cm/cm.cm is stable]
... lots more loading
[loading (sources.cm):src/frontend/(sources.cm):frontend.sml]
[New bindings added.]
val it = true : bool
\end{verbatim} 

Next, load the user extensions, again using
the compilation manager.

\begin{verbatim} 
- CM.make "../prover/sources.cm"; 
[scanning ../prover/sources.cm]
[loading ../prover/(sources.cm):../../sml/std-lib/lib.sig.sml]
[loading ../prover/(sources.cm):../../sml/std-lib/lib.sml]
[loading ../prover/(sources.cm):canonical_lf.sml]
[loading ../prover/(sources.cm):translate.sml]
[New bindings added.]
val it = true : bool
- 
\end{verbatim} 

Finally load whatever elf file you want to manipulate,
via [[Twelf.make]]

\begin{verbatim} 
- Twelf.make ("../sources.cfg");
[Opening file ../sources.cfg]
[Closing file ../sources.cfg]
[Opening file ../prop.elf]
prop : type.
top : prop.
bot : prop.
and : prop -> prop -> prop.
/\ : prop -> prop -> prop = [x:prop] [y:prop] and x y.
imp : prop -> prop -> prop.
... lots more signature
[Closing file ../prop.elf]
val it = OK : Twelf.Status
\end{verbatim} 

Now you should be able to use the modules exported
by the [[sources.cm]] file in the Twelf directory, 
(those between [[Library]] modules [[is]]),
along with whatever code you write.  In our case
the various translators for parsing and printing.

\subsection{Using Twelf}

Some common things you'll want to do:

\begin{itemize} 
\item Print a signature.

\begin{verbatim} 
- Twelf.Print.sgn();
prop : type.
top : prop.
...
\end{verbatim} 

\item Print a signature with your own printer.

\begin{verbatim} 
- Twelf.Print.Coq.sgn();
Definition and := (fun x => (fun y => (and x y))).
Definition or := (fun x => (fun y => (or x y))).
Definition imp := (fun x => (fun y => (imp x y))).
....
\end{verbatim} 

\item Get the size of the signature

\begin{verbatim} 
- IntSyn.sgnSize();
val it = (74,0) : IntSyn.cid * IntSyn.mid
\end{verbatim} 

\item Get the abstract syntax of an element (by number)
\begin{verbatim} 
- IntSyn.sgnLookup 5;
val it = ConDec ("or",NONE,0,Normal,Pi ((#,#),Pi #),Type) : IntSyn.ConDec
\end{verbatim} 

\item Get the whole signature
\begin{verbatim} 
- map IntSyn.sgnLookup (Lib.upto(0,73));
val it =
  [ConDec ("prop",NONE,0,Normal,Uni Type,Kind),
   ConDec ("top",NONE,0,Normal,Root (#,#),Type),
   ConDec ("bot",NONE,0,Normal,Root (#,#),Type),
   ConDec ("and",NONE,0,Normal,Pi (#,#),Type),
   ...
   ConDec ("iff",NONE,0,Normal,Pi (#,#),Type),...] : IntSyn.ConDec list
\end{verbatim} 

\end{itemize} 

Now that the sandbox is complete with toys, play away!


\bibliographystyle{abbrv}

\bibliography{all}

\end{document}
