%-------------------------------------------------------------------------------
% Topmatter                                                                     
%-------------------------------------------------------------------------------

\documentclass[11pt,twoside]{article}
\usepackage{noweb,fullpage,clrscode,proof,hyperref,amsmath,amssymb}
\usepackage[open-square,define-standard-theorems,roman-theorems]{QED}

%-------------------------------------------------------------------------------
% Title                                                                         
%-------------------------------------------------------------------------------

\title{Towards a Focusing Inverse-Method Theorem Prover for Canonical LF}
\author{Sean McLaughlin}

%-------------------------------------------------------------------------------
% Begin                                                                         
%-------------------------------------------------------------------------------

\begin{document} 
\maketitle
\arraycolsep=12pt

%---------------------------------  Constants  ---------------------------------

\newcommand{\Nil}{\mathtt{nil}}
\newcommand{\Kind}{\mathtt{kind}}
\newcommand{\Type}{\mathtt{type}}
\newcommand{\IdSub}{\mathtt{id}}
\newcommand{\Of}[2]{#1\ :\ #2}

%-----------------------------------  Terms  -----------------------------------

\newcommand{\PiTyp}[2]{\Pi #1.#2}
\newcommand{\Lam}[1]{\lambda #1}
\newcommand{\Appl}[2]{#1\ #2}
\newcommand{\Arr}[2]{#1\to #2}
\newcommand{\HSub}[4]{[#1/#2]^{#3}#4}

%--------------------------------  Separators  ---------------------------------

\newcommand{\Spb}{\ |\ }
\newcommand{\Shift}{\uparrow}
\newcommand{\Set}[1]{\{#1\}}
\newcommand{\Comp}{~\circ~}
\newcommand{\App}{~@~}
\newcommand{\SeqArr}{\Longrightarrow}
\newcommand{\LFArr}{\to}

%--------------------------------  Basic math  ---------------------------------

\newcommand{\set}[1]{\{#1\}}
\newcommand{\card}[1]{\left|#1\right|}
\newcommand{\Not}{\neg}
\newcommand{\And}{\wedge}
\newcommand{\Or}{\vee}
\newcommand{\AND}{\bigwedge}
\newcommand{\OR}{\bigvee}
\newcommand{\Imp}{\supset}
\newcommand{\Iff}{\Longleftrightarrow}
\newcommand{\Arr}{\Rightarrow}


\def\incomplete#1{XXX #1 XXX}






%-------------------------------------------------------------------------------
% Abstract                                                                      
%-------------------------------------------------------------------------------

\begin{abstract} 
This is an experiment at implementing \emph{Canonical LF}.  We intend
to present muliple forms of representation.  We will begin with a
simple name-carrying implementation of Canonical LF.  We will continue
with spine forms and explicit substitutions, and finally add
meta-variables.  Through this process, we hope to experiment with the
various representations to find a good trade-off between efficiency
and elegance.
\end{abstract} 

%-------------------------------------------------------------------------------
% Canonical LF                                                                  
%-------------------------------------------------------------------------------


%-------------------------------------------------------------------------------
% Canonical LF                                                                  
%-------------------------------------------------------------------------------

\section{Canonical LF}
We begin by describing Canonical LF, the language at the heart of our
work.  While the various representations will differ from the one presented here, 
(e.g., for efficiency) this language should always be kept in mind.

The main difference between Canonical LF and earlier versions
is the lack of a type annotation on $\lambda$-expressions.
See Harper and Licata\cite{HarperLicataCLF} for an extended
exposition.

$$
\begin{array}{llll}
\textbf{Kinds} & K & ::= & \Type \Spb \PiTyp{x}{A}{K} \\
\textbf{Canonical Type Families} & A & ::= & P \Spb \PiTyp{x}{A_2}{A} \\
\textbf{Atomic Type Families} & P & ::= & a \Spb \Appl{P}{M} \\
\textbf{Canonical Terms} & M & ::= & R \Spb \Lam{x}{M} \\
\textbf{Atomic Terms} & R & ::= &  x \Spb c \Spb \Appl{R}{M}\\
\textbf{Signatures} & \Sigma & ::= & \cdot \Spb \Sigma,\Of{c}{A} 
\Spb \Sigma,\Of{a}{K}\\
\textbf{Contexts} & \Gamma & ::= & \cdot \Spb \Gamma,\Of{x}{A}\\
% \textbf{Simple Types} & \alpha & ::= & a \Spb \Arr{\alpha_1}{\alpha_2} \\
\end{array}
$$

%-------------------------------------------------------------------------------
% Spine Form LF                                                                 
%-------------------------------------------------------------------------------

\subsection{Spine-Form Canonical LF}

There are a number of difficulties with the name-carrying 
representation\footnote{i.e., variable names associated with binders}
of Canonical LF.  The first is that we must
implement capture-avoiding substitution and $\alpha$-conversion,
a notoriously delicate and error-prone process.
We can circumvent this difficulty
by using DeBruijn indices\cite{DeBruijn80}. 

 A more significant 
difficulty lies in the implementation of hereditary substitution. 
\incomplete{citation needed}
When applying a substitution, we often need to determine whether
the head of an expression is a constant or a variable in order
to know which rule to apply.  Thus, for a term of the form
$$f\ x_1\ x_2\ \ldots\ x_n = (\ldots((f\ x_1)\ x_2)\ \ldots\ x_n) $$
we need to take apart $n$ applications just to determine how
a substitution should be applied.  Later, when we implement
unification, that algorithm will need to compare the heads
of such terms for equality.  Thus, quick access to the head
of such a term is essential for a reasonably efficient implementation.
We thus define \emph{Spine-Form Canonical LF}.

$$
\begin{array}{llll}
\mathbf{Kinds} & K & ::= & \Type \Spb \SpPiTyp{A}{K} \\
\mathbf{Canonical\ Type\ Families} & A & ::= & P \Spb \SpPiTyp{A_1}{A_2} \\
\mathbf{Atomic\ Type\ Families} & P & ::= & a\cdot S \\
\mathbf{Canonical\ Terms} & M & ::= & R \Spb \SpLam{M} \\
\mathbf{Atomic\ Terms} & R & ::= & H\cdot S \\
\mathbf{Heads} & H & ::= & c \Spb i\\
\mathbf{Spines} & S & ::= & \Nil \Spb M;S\\
\end{array} 
$$

In the following judgments will have the same
form for different classes.  For instance,
the rules for $\Pi$-types and $\Pi$-kinds will
oftentimes be identical in structure.  To avoid the
repetition of rules, we introduce a convenient 
syntax.

$$
\begin{array}{llll}
\mathbf{Levels} & L & ::= & \Type \Spb \Kind \\
\mathbf{Expressions} & U & ::= & L \Spb \SpPiTyp{U_1}{U_2} \Spb \lambda U \Spb H\cdot S \\
\mathbf{Heads} & H & ::= & c \Spb i\\
\mathbf{Spines} & S & ::= & \Nil \Spb U;S\\
\end{array} 
$$

Constants $c$ are either type ($a$) or term ($c$) constants.
The rules that follow will refer to this basic syntax.  While this
is somewhat less precise than the more explicit separation of 
levels in the syntax above (indeed, we can easily write gramatically
correct nonsense in this language, such as $\SpLam{(\SpPiTyp{U1}{U2})}$)
we are willing to allow such terms to lessen the number of rules.
Terms that are not even gramatically correct (much less type-correct) 
in the original language, will be excluded by type-checking (rather than expression 
formation, as in the previous version).

We will freely mix the more concrete classes such as $K,A$ and $P$
above when the rules restrict the expressions to such cases.
We see no potential for confusion however, and again notational
convenience is our guide.  




%-------------------------------------------------------------------------------
% Typechecking                                                                  
%-------------------------------------------------------------------------------

While we have just described Canonical LF, a language where all
expressions are canonical, it is sometimes useful to relax the canonicity
restriction to take advantage of various implementation 
techniques.  To get a better idea of the efficiency of various
implementations, we present here two different methods of
typechecking, called \emph{eager} and \emph{lazy}, with
two minor variants of each method.


\subsection{Eager Typechecking}

\renewcommand{\PiTyp}[2]{\Pi #1.#2}
\renewcommand{\Lam}[1]{\lambda #1}
\newcommand{\CheckTy}[3][\Gamma]{#1\vdash #2 \Leftarrow #3}
\newcommand{\Focus}[4][\Gamma]{#1\vdash #2 : #3 > #4}
\newcommand{\Equiv}[2]{#1\equiv #2}

%-------------------------------------------------------------------------------
% Terms                                                                         
%-------------------------------------------------------------------------------

\subsection{Terms}

$$
\begin{array}{llll}
\mathbf{Kinds} & K & ::= & \Type \Spb \PiTyp{A}{K} \\
\mathbf{Types} & A & ::= & a\cdot S \Spb \PiTyp{A_1}{A_2} \\
\mathbf{Terms} & M & ::= & H\cdot S \Spb \Lam{M} \\
\mathbf{Heads} & H & ::= & c \Spb i\\
\mathbf{Spines} & S & ::= & \Nil \Spb M;S\\
\end{array} 
$$


%-------------------------------------------------------------------------------
% Typecheck                                                                     
%-------------------------------------------------------------------------------

\bigskip 
\framebox{$\CheckTy{K}{\Kind}$}
\bigskip 

$$
\begin{array}{cc}
\infer{\CheckTy{\Type}{\Kind}}{} &
\infer{\CheckTy{\PiTyp{A}{K}}{\Kind}}{\CheckTy{A}{\Type} & \CheckTy[\Gamma,A]{K}{\Kind}}
\end{array} 
$$

\bigskip 
\framebox{$\CheckTy{A}{\Type}$}
\bigskip 

$$
\begin{array}{cc}
\infer{\CheckTy{\PiTyp{A_1}{A_2}}{\Type}}{\CheckTy{A_1}{\Type} & \CheckTy[\Gamma,A_1]{A_2}{\Type}} &
\infer{\CheckTy{a\cdot S}{\Type}}{\Sigma(a) = K & \Focus{S}{K}{\Type}}
\end{array} 
$$

\bigskip 
\framebox{$\CheckTy{M}{A}$}
\bigskip 

We assume that if $\CheckTy{M}{A}$ then $\CheckTy{A}{\Type}$.

$$
\begin{array}{cc}
\infer{\CheckTy{\Lam{M}}{\PiTyp{A_1}{A_2}}}{\CheckTy[\Gamma,A_1]{M}{A_2}} &
\infer{\CheckTy{c\cdot S}{A_2}}{\Sigma(c)=A_1 & \Focus{S}{A_1}{A_2'} & \Equiv{A_2'}{A_2}} \\\\
\multicolumn{2}{c}{\infer{\CheckTy{i\cdot S}{A_2}}{\Gamma(i)=A_1 & \Focus{S}{A_1[\Shift^i]}{A_2'} & \Equiv{A_2'}{A_2}}}
\end{array} 
$$

\bigskip 

Note that you must shift the type you extract from $\Gamma$, as the
free variables (indices) should point to the slots before $i$.  Moving
the type $A$ from the context to the consequent must adjust the pointers.

\bigskip 
\framebox{$\Focus{S}{K}{\Type}$}
\bigskip 

$$
\begin{array}{lr}
\infer{\Focus{\Nil}{\Type}{\Type}}{} & 
\infer{\Focus{(M;S)}{\PiTyp{A}{K}}{\Type}}{\CheckTy{M}{A} & \Focus{S}{K[M\cdot\IdSub]}{\Type}}
\end{array} 
$$

\bigskip 
\framebox{$\Focus{S}{A_1}{A_2}$}
\bigskip 

Again, we assume that if $\Focus{S}{A_1}{A_2}$ that $\CheckTy{A_1}{\Type}$.

$$
\begin{array}{lr}
\infer{\Focus{\Nil}{a\cdot S}{a\cdot S}}{} & 
\infer{\Focus{(M;S)}{\PiTyp{A_1}{A_2}}{A_3}}{\CheckTy{M}{A_1} & \Focus{S}{A_2[M\cdot\IdSub]}{A_3}}
\end{array} 
$$

%-------------------------------------------------------------------------------
% Substitutions                                                                 
%-------------------------------------------------------------------------------

\subsection{Substitutions}

Even though the substitution judgment (at this point) is operational,
and in principle a substitution is simply a list of terms $M$,
it is still useful to define a syntax of substitutions.  

\newcommand{\Msub}{[\sigma]}
\newcommand{\Ssub}{[1\cdot(\sigma\Comp\Shift)]}

$$
\begin{array}{llll}
\mathbf{Substitutions} & \sigma & ::= & M\cdot\sigma \Spb \Shift^n \\
\end{array} 
$$

(The notation $\Shift$ means $\Shift^1$, and $\IdSub$ means $\Shift^0$.)  

We can \emph{apply} subsitutions to terms.

\bigskip
\framebox{$K\Msub = K'$}

\begin{align*} 
\Type\Msub &= \Type \\
(\PiTyp{A}{K})\Msub &= \PiTyp{(A\Msub)}{(K\Ssub)}\\
\end{align*} 

\framebox{$A\Msub = A'$}

\begin{align*} 
(a\cdot S)\Msub &= a\cdot (S\Msub) \\
(\PiTyp{A_1}{A_2})\Msub &= \PiTyp{(A_1\Msub)}{(A_2\Ssub)} \\
\end{align*} 

\framebox{$M\Msub = M'$}

\begin{align*} 
(\Lam{M})\Msub &= \Lam{(M\Ssub)}\\
(c\cdot S)\Msub &= c\cdot S\Msub \\
(i\cdot S)\Msub &= \begin{cases}
                     j\cdot S\Msub \mbox{\ if $i\Msub = j$} \\
                     M \App S\Msub\mbox{\ if $i\Msub = M$}
                   \end{cases} 
\end{align*} 

\framebox{$S\Msub = S'$}

\begin{align*} 
\Nil\Msub &= \Nil\\
(M;S)\Msub &= M\Msub;S\Msub
\end{align*} 

\framebox{$i\Msub = M$}

\begin{align*} 
1[M\cdot\sigma] &= M\\
n+1[M\cdot\sigma] &= n[\sigma]\\
i[\Shift^n] &= i+n\\
% i[\sigma_1\Comp\sigma_2] &= (i[\sigma_1])[\sigma_2]
\end{align*} 

We still need the notion of beta reduction when a 
head gets instantiated with a lambda.  We show
only the possible cases.

\bigskip 
\framebox{$M \App S = M'$}

\begin{align*} 
(H\cdot S)\App\Nil &= H\cdot S\\
\Lam{M}\App(M';S) &= M[M'\cdot\IdSub]\App S
\end{align*} 

The rules for composing substitutions are:

\bigskip 
\framebox{$\sigma\Comp\sigma' = \sigma''$}

$$
\begin{array}{llll}
(M\cdot \sigma) & \Comp \sigma' &= &M[\sigma']\cdot (\sigma\Comp\sigma') \\
\Shift^n & \Comp \Shift^m &= &\Shift^{n+m}\\
\Shift^0 & \Comp \sigma &= &\sigma\\
\Shift^{n+1}&\Comp (M\cdot\sigma) &= &\Shift^n\Comp\sigma
\end{array} 
$$

%-------------------------------------------------------------------------------
% Equivalence                                                                   
%-------------------------------------------------------------------------------

\subsection{Equivalence} 

If we only allowed constant declarations in a signature then checking equivalence
of terms would be a simple matter of checking syntactic equality.  
With definitions of the form $c : A = M$, we must account
for the fact that a focusing phase might return a type $A$ to 
check against a type $A'$ that are not syntactically equal, but
if one expanded all the definitions and normalized the resulting
terms than they would be identical.  We thus need a judgment for the
equivalence of types $A$ and terms $M$.  (Since we are not allowing
type level definitions, we do not need to check for equivalent kinds.)

\newcommand{\StepsTo}{\rightsquigarrow}
\newcommand{\NoStep}{\not\rightsquigarrow}

We use the judgments $c\StepsTo M$  to mean
that the constant $c$ has definition $M$, and $c\NoStep$ means
$c$ does not have a definition (i.e. is a declared constant).

\bigskip 
\framebox{$\Equiv{A}{A'}$}
\bigskip 

$$
\begin{array}{lcr}\
\infer{\Equiv{A}{A'}}{\Equiv{A'}{A}} &  
\infer{\Equiv{\PiTyp{A_1}{A_2}}{\PiTyp{A_1'}{A_2'}}}{\Equiv{A_1}{A_1'} & \Equiv{A_2}{A_2'}} & 
\infer{\Equiv{a\cdot S}{a\cdot S'}}{\Equiv{S}{S'}} 
\end{array} 
$$

\bigskip 
\framebox{$\Equiv{S}{S'}$}
\bigskip 

$$
\begin{array}{lcr}
\infer{\Equiv{\Nil}{\Nil}}{} &
\infer{\Equiv{M;S}{M';S'}}{\Equiv{M}{M'} & \Equiv{S}{S'}}
\end{array} 
$$

\bigskip 
\framebox{$\Equiv{M}{M'}$}
\bigskip 

$$
\begin{array}{cccc}
\infer{\Equiv{\Lam{M}}{\Lam{M'}}}{\Equiv{M}{M'}} &
\infer{\Equiv{i\cdot S}{i\cdot S'}}{\Equiv{S}{S'}} &
\infer{\Equiv{c\cdot S}{c\cdot S'}}{\Equiv{S}{S'}} &
\infer{\Equiv{c\cdot S}{M}}{c\StepsTo M' & \Equiv{M'@S}{M}} 
\end{array} 
$$


\section{Lazy Typechecking}

There are two variations on lazy typechecking, with the
same distinction as the eager case, viz. the existence
of a syntactic composition of substitutions.

In the lazy case we relax the canonicity requirements and 
admit terms into the language with explicit $\beta$-redexes
and explicit substitutions.

%-------------------------------------------------------------------------------
% Terms                                                                         
%-------------------------------------------------------------------------------

\subsection{Terms}

$$
\begin{array}{llll}
\mathbf{Levels} & L & ::= & \Type \Spb \Kind \\
\mathbf{Expressions} & U,V & ::= & L \Spb \PiTyp{U_1}{U_2} \Spb \lambda U \Spb H\cdot S \Spb U[\sigma] \Spb M @ S\\
\mathbf{Heads} & H & ::= & c \Spb i\\
\mathbf{Spines} & S & ::= & \Nil \Spb U;S \Spb S[\sigma]\\
\mathbf{Eager\ Substitutions} & \sigma & ::= & M\cdot\sigma \Spb \Shift^n \\
\mathbf{Lazy\ Substitutions} & \sigma & ::= & M\cdot\sigma \Spb \Shift^n \Spb \sigma_1 \Comp \sigma_2\\
\end{array} 
$$

%-------------------------------------------------------------------------------
% Weak Head Normal Form                                                         
%-------------------------------------------------------------------------------

\subsection{Weak Head Normalization}

For terms With explicit substitutions $U[\sigma]$ and redexes $M @ S$,
we can usually not decide which rule to apply during typechecking.
We need to discover the top level structure of the term.  
The least work we need to do to discover this form is called 
\emph{weak head normalization}.  


\bigskip 
\framebox{$U\Weak U'$}
\bigskip 

$$
\begin{array}{cc}
\infer{L \Weak L}{} &
\infer{\PiTyp{U_1}{U_2} \Weak \PiTyp{U_1}{U_2}}{} \\\\
\infer{\Lam{U} \Weak \Lam{U}}{} &
\infer{H\cdot S \Weak H\cdot S}{} \\\\
\infer{L[\sigma] \Weak L}{}&
\infer{(\PiTyp{U_1}{U_2})[\sigma] \Weak \PiTyp{(U_1[\sigma])}{(U_2\Ssub)}}{} \\\\
\infer{(\Lam{U})[\sigma] \Weak \Lam{(U\Ssub)}}{} &
\infer{(c\cdot S)[\sigma] \Weak c\cdot (S[\sigma])}{} \\\\
\infer{(U[\sigma])[\sigma'] \Weak U'}{U[\sigma\Comp\sigma']\Weak U'} \\\\
\infer{M @ S \Weak H\cdot S}{M\Weak H\cdot S & S \Weak \Nil} &
\infer{M @ S \Weak M_3}{M\Weak \Lam{M_1} & S \Weak M_2;S' & M_1[M_2\cdot\IdSub] @ S'\Weak M_3} \\\\
\infer{(i\cdot S)[\sigma]\Weak j\cdot (S[\sigma])}{i[\sigma] \Weak j} &
\infer{(i\cdot S)[\sigma]\Weak M'}{i[\sigma] \Weak M & M@(S[\sigma])\Weak M'} \\\\
\end{array} 
$$

\bigskip 
\framebox{$S\Weak S'$}
\bigskip 

$$
\begin{array}{cc}
\infer{\Nil \Weak \Nil}{} &
\infer{M;S \Weak M;S}{} \\\\
\infer{\Nil[\sigma] \Weak \Nil}{} &
\infer{(M;S)[\sigma] \Weak M[\sigma];S[\sigma]}{} 
\end{array} 
$$

\bigskip 
\framebox{$i[\sigma] \Weak M$}
\bigskip 

$$
\begin{array}{cc}
\infer{i[\Shift^n] \Weak i + n}{} &
\infer{1[M;S] \Weak M'}{M\Weak M'} \\\\
\infer{(n+1)[M\cdot \sigma] \Weak M'}{n[\sigma]\Weak M'} &
\infer[(*)]{i[\sigma_1\Comp\sigma_2] \Weak M'}{i[\sigma_1] \Weak M & M[\sigma_2]\Weak M'} 
\end{array} 
$$

%-------------------------------------------------------------------------------
% Typecheck                                                                     
%-------------------------------------------------------------------------------

\subsection{Typechecking}

Typechecking is slightly more complicated in the lazy case, as
we need to return any terms that have been partially evaluated
by the typechecking process.

\bigskip 
\framebox{$\LzCheckTy{U}{V}{\Gamma'}{U'}{V'}$}
\bigskip 

$$
\begin{array}{c}
\infer{\LzCheckTy{U}{V}{\Gamma'}{U''}{V''}}{U\Weak U' & V\Weak V' & \LzCheckTy{U'}{V'}{\Gamma'}{U''}{V''}}\\\\
\infer{\LzCheckTy{\Type}{\Kind}{\Gamma}{\Type}{\Kind}}{} \\\\
\infer{\LzCheckTy{\PiTyp{A}{U}}{L}{\Gamma''}{\PiTyp{A''}{U'}}{L}}{\LzCheckTy{A}{\Type}{\Gamma'}{A'}{\Type} & \LzCheckTy[\Gamma',A']{U}{L}{\Gamma'',A''}{U'}{L}} \\\\
% \infer{\LzCheckTy{U[\sigma]}{V}{\Gamma'}{U''}{V'}}{U[\sigma]\Weak U' & \LzCheckTy{U'}{V}{\Gamma'}{U''}{V'}} \\\\
% \infer{\LzCheckTy{M @ S}{A}{\Gamma'}{M''}{A'}}{M@S\Weak M' & \LzCheckTy{M'}{A}{\Gamma'}{M''}{A'}} \\\\
\infer{\LzCheckTy{\Lam{M}}{A}{\Gamma'}{\Lam{M'}}{\PiTyp{A_1'}{A_2'}}}{A\Weak \PiTyp{A_1}{A_2} & \LzCheckTy[\Gamma,A_1]{M}{A_2}{\Gamma',A_1'}{M'}{A_2'}} \\\\
\infer{\LzCheckTy{c\cdot S}{V_1}{\Gamma'}{c\cdot S'}{V_1'}}{\Sigma(c) = V_2 & \LzFocus{S}{V_2}{\Gamma'}{S'}{V_2} & \LzEquiv{V_2}{V_1}{V_2}{V_1'}} \\\\
\infer{\LzCheckTy{i\cdot S}{A_1}{\Gamma'}{i\cdot S'}{A_1'}}{\Gamma(i) = A_2 & \LzFocus{S}{A_2}{\Gamma'}{S'}{A_2} & \LzEquiv{A_2}{A_1}{A_2}{A_1'}} \\\\
\end{array} 
$$

Note that in the rule for $c\cdot S$, $\Sigma$ contains only canonical terms,
thus the weak head normalization does not affect it.
Also recall note \ref{context:shift}.

\bigskip 
\framebox{$\LzFocus{S}{U}{\Gamma'}{S'}{U'}$}
\bigskip 

$$
\begin{array}{c}
\infer{\LzFocus{U}{V}{\Gamma'}{U''}{V''}}{U\Weak U' & V\Weak V' & \LzFocus{U'}{V'}{\Gamma'}{U''}{V''}}\\\\
\infer{\LzFocus{\Nil}{\Type}{\Gamma}{\Nil}{\Type}}{} \\\\
\infer{\LzFocus{P}{L}{\Gamma}{P}{L}}{} \\\\
\infer{\LzFocus{M;S}{\PiTyp{A}{V}}{\Gamma''}{M';S'}{\PiTyp{A'}{V'}}}{\LzCheckTy{M}{A}{\Gamma'}{M'}{A'} & \LzFocus[\Gamma']{S}{V[M'\cdot\IdSub]}{\Gamma''}{S'}{V'}} \\\\
\end{array} 
$$

The rules for evaluating composition of substitutions $\sigma_1\Comp\sigma_2$
are the same as for the eager case, though $M[\sigma]$ should be read
as the lazy explicit substitution.

%-------------------------------------------------------------------------------
% Equivalence                                                                   
%-------------------------------------------------------------------------------

\subsection{Equivalence} 

Note that we do not consider equivalence of substitutions or redexes, but rather
immediately consider the weak head normal form.  Would it be interesting
to try equivalence there?

\bigskip 
\framebox{$\LzEquiv{U_1}{U_2}{U_1'}{U_2'}$}
\bigskip 

$$
\begin{array}{c}
\infer{\LzEquiv{U_1}{U_2}{U_1''}{U_2''}}{U_1\Weak U_1' & U_2\Weak U_2' & \LzEquiv{U_1'}{U_2'}{U_1''}{U_2''}}\\\\
\infer{\LzEquiv{U_1}{U_2}{U_1'}{U_2'}}{\LzEquiv{U_2}{U_1}{U_2'}{U_1'}} \\\\
\infer{\LzEquiv{\PiTyp{U_1}{U_2}}{\PiTyp{U_3}{U_4}}{\PiTyp{U_1'}{U_2'}}{\PiTyp{U_3'}{U_4'}}}{\LzEquiv{U_1}{U_3}{U_1'}{U_3'} & \LzEquiv{U_2}{U_4}{U_2'}{U_4'}} \\\\
\infer{\LzEquiv{c\cdot S_1}{c\cdot S_2}{c\cdot S_1'}{c\cdot S_2'}}{\LzEquiv{S_1}{S_2}{S_1'}{S_2'}} \\\\
\infer{\LzEquiv{i\cdot S_1}{i\cdot S_2}{i\cdot S_1'}{i\cdot S_2'}}{\LzEquiv{S_1}{S_2}{S_1'}{S_2'}} \\\\
\infer{\LzEquiv{\Lam{M_1}}{\Lam{M_2}}{\Lam{M_1'}}{\Lam{M_2'}}}{\LzEquiv{M_1}{M_2}{M_1'}{M_2'}} \\\\
% \infer{\LzEquiv{U_1[\sigma_1]}{U_2[\sigma_2]}{U_1'[\sigma_1']}{U_2'[\sigma_2']}}{\LzEquiv{U_1}{U_2}{U_1'}{U_2'} & \LzEquiv{\sigma_1}{\sigma_2}{\sigma_1'}{\sigma_2'}} \\\\
\infer{\LzEquiv{c\cdot S}{M}{M''}{M'}}{c\StepsTo M' & \LzEquiv{M'@S}{M}{M''}{M'}} 
\end{array} 
$$

\bigskip 
\framebox{$\LzEquiv{S_1}{S_2}{S_1'}{S_2'}$}
\bigskip 

$$
\begin{array}{c}
\infer{\LzEquiv{S_1}{S_2}{S_1''}{S_2''}}{S_1\Weak S_1' & S_2\Weak S_2' & \LzEquiv{S_1'}{S_2'}{S_1''}{S_2''}}\\\\
\infer{\LzEquiv{\Nil}{\Nil}{\Nil}{\Nil}}{}\\\\
\infer{\LzEquiv{M_1;S_1}{M_2;S_2}{M_1';S_1'}{M_2';S_2'}}{\LzEquiv{M_1}{M_2}{M_1'}{M_2'} & \LzEquiv{S_1}{S_2}{S_1'}{S_2'}} \\\\
% \infer{\LzEquiv{S_1[\sigma_1]}{S_2[\sigma_2]}{S_1'[\sigma_1']}{S_2'[\sigma_2']}}{\LzEquiv{S_1}{S_2}{S_1'}{S_2'} & \LzEquiv{\sigma_1}{\sigma_2}{\sigma_1'}{\sigma_2'}} \\\\
\end{array} 
$$



%-------------------------------------------------------------------------------
% The Inverse Method                                                            
%-------------------------------------------------------------------------------


\section{The Inverse Method}

\subsection{Requirements}

\begin{enumerate} 
\item Typechecker
  \begin{itemize} 
  \item Internal api, e.g. ``\%doublecheck''
  \item External api, for Deepak say.  Perhaps support explicit sharing of terms.
  \end{itemize} 
\item Term Reconstruction
\item Logic programming engine (Bottom up)
\item M2 Checker (\%mode,\%worlds,\%terminates,\%covers,\%total)
\item logic programming engine (Top down)
\item Meta-theorem prover (produce .prf files from .thm files)
\end{enumerate} 

\subsection{The Subformula Property}

\begin{Theorem} The cut rule is admissible in LF \end{Theorem} 
\XXX{Needs reference}

It is a property of cut-free sequent calculi that all propositions occurring
in a derivation are \emph{subformulas} of the endsequent in the following sense.

We define \emph{signed subformulas} of a formula $F$ with judgments:

\begin{align*} 
\PSubform{A} & & \mbox{$A$ is a positive subformula of the initial goal.}\\
\NSubform{A} & & \mbox{$A$ is a negative subformula of the initial goal.}
\end{align*} 

where 
$$\Psi ::= \circ \Spb \Psi, v:A \Spb \Psi, x:A$$

Here $v$ ranges over arbitrary terms and $x$ ranges over
fixed parameters.


These rules are defined as a forward logic program by the following inference
rules:

\bigskip 
\begin{tabular}{cc}
\infer{\NSubform{A}}{\PSubform{A\LFArrow B}} &
\infer{\NSubform{B}}{\PSubform{A\LFArrow B}} \\
\infer{\PSubform{A}}{\NSubform{A\LFArrow B}} &
\infer{\PSubform{B}}{\NSubform{A\LFArrow B}} \\
\infer{\PSubform[\Psi,x:A]{B}}{\PSubform{\PiTyp{x}{A}{B}}} &
\infer{\NSubform[\Psi,u:A]{B}}{\NSubform{\PiTyp{u}{A}{B}}} \\
\end{tabular} 
\bigskip 

Note that there are no rules for families of kind $\Type$.

To find the subformulas of a given goal formula, we run
the above rules as a forward logic program to saturation.  

\subsection{Admissible Substitutions}

Explicit substitutions are an integral part of the theory and 
implementation of our development.  They are necessary because
when logic variables arise during proof search, it is not always
possible to apply a substitution immediately.  While contexts 
do the necessary bookkeeping of tracking the types of bound variables,
substitutions track variable (term) instantiations.
The language of substitutions is

$$\sigma ::= \circ \Spb \sigma,(M/u) \Spb \sigma, (y//x)$$ 

where $M$ ranges over arbitrary terms, and $y$ ranges over
parameters.

An invariant(*) of admissible substitutions is that if $\AdmSub{\sigma}{\Psi}$ then
the domain of $\sigma$ is exactly the free variables of $\Psi$.  The codomain
need not be all of $\Sigma$.  Similarly, whenever we apply 
$A[\Theta]$ it will be the case that $\Theta$ substitutes exactly for the
free variables of $A$.  

The judgment on admissible substitutions will be 
$$\AdmSub{\sigma}{\Psi}$$
read ``$\sigma$ is an admissible substitution from $\Psi$ to $\Sigma$''.
and meaning that $\sigma$ substitutes for all the free variables in
$\Psi$, but can mention variables in $\Sigma$.  

\bigskip 

\infer{\AdmSub{\cdot}{\cdot}}{} 
\bigskip 

\infer{\AdmSub{(\sigma,M/u)}{\Psi,u:A}}{\AdmSub{\sigma}{\Psi} & \CheckTy[\Sigma]{M}{A[\sigma]}} 
\bigskip 

\infer{\AdmSub{(\sigma,y//x)}{\Psi,x:A}}{\AdmSub{\sigma}{\Psi} & \SynthTy[\Sigma]{y}{A'} & A' = \HerSub{A}} 
\bigskip 

\subsection{Forward Sequents}

For the inverse method, we think of the usual sequent calculus rules, which work bottom-up, 
backwards to work top-down.  

The judgment 

$$\Sequent{\Sigma}{A_1[\tau_1],\ldots,A_n[\tau_n]}{C[\sigma]}$$

where for all $i$, 

\begin{align*} 
  &\AdmSub{\tau_i}{\Psi_i}\\
  &\NSubform[\Psi_i]{A_i}\\
  &\AdmSub{\sigma}{\Psi}\\
  &\PSubform[\Psi]{C}
\end{align*} 

can be read as (modulo constraints on substitutions)
``Given with parameters $\Sigma$ and hypotheses and derived formulas 
$A_i$ under admissible substitutions $\tau_i$, the goal $C$ under admissible
substitution $\sigma$ is derivable.''

\subsection{Subsumption}

$$
\Sequent{\Sigma}{A_1[\tau_1],\ldots,A_n[\tau_n]}{C[\sigma]} \leq 
\Sequent{\Sigma'}{A_1[\tau_1'],\ldots,A_k[\tau_k']}{C[\sigma']}
$$

if there exists an admissible substitution $\AdmSub[Sigma']{\Theta}{\Sigma}$
such that 
$\tau_1' = \tau_1 \Comp \Theta, \ldots, \tau_k' = \tau_k \Comp \Theta, \sigma' = \sigma \Comp \Theta$

for some $n\leq k$.  

\subsection{Unification}

\subsection{Most General Unifiers}
In the pattern fragment of LF, most general unifiers exist.  Therefore the
function $\mgu(x,y)$ is well defined.  Keeping our invariant (*) in mind, 
we will write $\mgu = (\Theta_1,\Theta_2)$ when the domain of the intended
terms to which we apply the substitutions have different sets of free variables.
That is, if the most general unifier of $A_1,A_2$ is $\Theta$ and 
$fv(A_1) \neq fv(A_2)$ then 
\begin{align*} 
  \Theta_1 &= \mathbf{filter}\ (\lambda(x,\_).\ \mathbf{mem}(fv(A_1,x),x))\ \Theta \\ 
  \Theta_2 &= \mathbf{filter}\ (\lambda(x,\_).\ \mathbf{mem}(fv(A_2,x),x))\ \Theta
\end{align*} 


\subsection{Forward Sequent Rules, Pattern Fragment}

\infer[\textbf{Init}]{\Sequent{\Sigma}{P_1[\Theta_1]}{P_2[\Theta_2]}}{}
\bigskip 
for any $\NSubform[\Psi_1]{P_1}, \PSubform[\Psi_2]{P_2}$
such that $\MGU{(\Theta_1,\Theta_2)}{\NSubform[\Psi_1]{P_1}}{\PSubform[\Psi_2]{P_2}}$
with $\AdmSub{\Theta_1}{\Psi_1}$ and $\AdmSub{\Theta_2}{\Psi_2}$.

\bigskip 
\infer[\textbf{Contract}]{\Sequent{\Sigma'}{\Gamma[\Theta],A[\tau_1 \Comp \Theta}{C[\sigma \Comp \Theta}}
                         {\Sequent{\Sigma}{\Gamma,A[\tau_1],A[\tau_2]}{C[\sigma]}}

\bigskip 
when $\AdmSub{\tau_1}{\Psi}$, $\AdmSub{\tau_2}{\Psi}$, and $\MGU[\Sigma']{\Theta}{\tau_1}{\tau_2}$
\bigskip 

\infer[\textbf{ArrowR1}]{\Sequent{\Sigma'}{\Gamma[\Theta]}{(A\LFArrow B)[\tau \Comp \Theta]}}
                       {\Sequent{\Sigma}{\Gamma,A[\tau]}{B[\sigma]}}
\bigskip 
when $\MGU[\Sigma']{\Theta}{\tau}{\sigma}$, $\AdmSub{\Theta}{\Psi}$, $\AdmSub{\sigma}{\Psi}$, 
and $\PSubform{A \LFArrow B}$
\bigskip 

\infer[\textbf{ArrowR2}]{\Sequent{\Sigma}{\Gamma}{(A\LFArrow B)[\sigma]}}
                        {\Sequent{\Sigma}{\Gamma}{B[\sigma]}}
\bigskip 
when $\AdmSub{\sigma}{\Psi}$, and $\PSubform{A \LFArrow B}$
\bigskip 

\infer[\textbf{ArrowL}]{\Sequent{\Sigma}{\Gamma_1[\Theta_1],\Gamma_2[\Theta_2],(A\LFArrow B)[\tau_1 \Comp \Theta_1]}{C[\sigma \Comp \tau_2]}}
                       {\Sequent{\Sigma_1}{\Gamma_1}{A[\tau_1]} & \Sequent{\Sigma_2}{\Gamma_2,B[\tau_2]}{C[\sigma]}}
\bigskip 
when
\begin{align*} 
  &\MGU{(\Theta_1,\Theta_2)}{\AdmSub{\Theta_1}{\Sigma_1}}{\AdmSub{\Theta_2}{\Sigma_2}}\\
  &\AdmSub{\tau_1}{\Psi}\\
  &\AdmSub{\tau_2}{\Psi}\\
  &\NSubform{A \LFArrow B}
\end{align*} 
\bigskip 

\infer[\textbf{PiL}]{\Sequent{\Sigma}{\Gamma, (\PiTyp{u}{A}{B})[\tau]}{C[\sigma]}}
                    {\Sequent{\Sigma}{\Gamma, B[\tau,M/u]}{C[\sigma]}}
\bigskip 
when $\NSubform{\PiTyp{u}{A}{B}}$

\bigskip 
\infer[\textbf{PiR}]{\Sequent{\Sigma}{\Gamma}{(\PiTyp{x}{A}{B})[\sigma]}}
                    {\Sequent{\Sigma,y:A'}{\Gamma}{B[\sigma,y//x]}}
\bigskip 
when $y\not\in \Gamma$, $A' = A[\sigma]$ and $\PSubform{\PiTyp{x}{A}{B}}$

\subsection{Junk}


\newcommand{\init}{(\Gamma_0^-,C_0^+)}

\infer[$Atom$]{P \vdash P}{}
\bigskip 
when $P \leq^+ \init$ and $P \leq^- \init$
\bigskip 

\infer[$Contract$]{\Gamma,A \vdash B}{\Gamma,A,A\vdash B}
\bigskip 

\infer[\to$R1$]{\Gamma\vdash A \LFArrow B}{\Gamma,A \vdash B & (A\LFArrow B \leq^+ \init)}
\bigskip 

\infer[\LFArrow$R2$]{\Gamma\vdash A \LFArrow B}{\Gamma \vdash B & A \not\in\Gamma & (A\LFArrow B \leq^+ \init)}
\bigskip 

\infer[\LFArrow$L$]{\Gamma_1,\Gamma_2,A \LFArrow B\vdash C}{\Gamma_1,\vdash A,& \Gamma_2,A \vdash B & (A\LFArrow B \leq^- \init)}
\bigskip 

\subsection{The First Order Case}

\newcommand{\seq}[3][\Sigma]{#1 ; #2 \vdash #3}
\newcommand{\unif}[3][\Sigma]{#1 ; #2 = #3}
\newcommand{\renaming}[1]{#1\ \mathsf{renaming}}

General Form:

$$\Sigma; \Gamma[\tau] \vdash C[\sigma]$$

which is shorthand for

$$\Sigma; A_1[\tau_1],\ldots,A_n[\tau_n] \vdash C[\sigma]$$

\bigskip 

\infer[$Atom$]{\seq{P_1^-[\rho\circ\tau]}{P_2^+[\tau]}} {}
\bigskip 
when $\unif{P_1^-[\rho\circ\tau]}{P_2^+[\tau]}$ and $\renaming{\rho}$
\bigskip 

\infer[\Pi$-R$]{\seq{\Gamma}{}}{}
\bigskip 



%-------------------------------------------------------------------------------
% Appendices                                                                    
%-------------------------------------------------------------------------------

%% \subsection{Twelf Grammar}

see \texttt{twelf/src/lambda/intsyn.sig}

\begin{verbatim} 

  type cid = int			(* Constant identifier        *)
  type mid = int                        (* Structure identifier       *)
  type csid = int                       (* CS module identifier       *)

  datatype 'a Ctx =			(* Contexts                   *)
    Null				(* G ::= .                    *)
  | Decl of 'a Ctx * 'a			(*     | G, D                 *)

  datatype Depend =                     (* Dependency information     *)
    No                                  (* P ::= No                   *)
  | Maybe                               (*     | Maybe                *)
  | Meta				(*     | Meta                 *)

  datatype Uni =			(* Universes:                 *)
    Kind				(* L ::= Kind                 *)
  | Type				(*     | Type                 *)

  datatype Exp =			(* Expressions:               *)
    Uni   of Uni			(* U ::= L                    *)
  | Pi    of (Dec * Depend) * Exp	(*     | Pi (D, P). V         *)
  | Root  of Head * Spine		(*     | H @ S                *)
  | Redex of Exp * Spine		(*     | U @ S                *)
  | Lam   of Dec * Exp			(*     | lam D. U             *)
  | EVar  of Exp option ref * Dec Ctx * Exp * (Cnstr ref) list ref
                                        (*     | X<I> : G|-V, Cnstr   *)
  | EClo  of Exp * Sub			(*     | U[s]                 *)
  | AVar  of Exp option ref             (*     | A<I>                 *)

  | FgnExp of csid * FgnExp             (*     | (foreign expression) *)

  | NVar  of int			(*     | n (linear, 
                                               fully applied variable
                                               used in indexing       *)

  and Head =				(* Head:                      *)
    BVar  of int			(* H ::= k                    *)
  | Const of cid			(*     | c                    *)
  | Proj  of Block * int		(*     | #k(b)                *)
  | Skonst of cid			(*     | c#                   *)
  | Def   of cid			(*     | d (strict)           *)
  | NSDef of cid			(*     | d (non strict)       *)
  | FVar  of string * Exp * Sub		(*     | F[s]                 *)
  | FgnConst of csid * ConDec           (*     | (foreign constant)   *)

  and Spine =				(* Spines:                    *)
    Nil					(* S ::= Nil                  *)
  | App   of Exp * Spine		(*     | U ; S                *)
  | SClo  of Spine * Sub		(*     | S[s]                 *)

  and Sub =				(* Explicit substitutions:    *)
    Shift of int			(* s ::= ^n                   *)
  | Dot   of Front * Sub		(*     | Ft.s                 *)

  and Front =				(* Fronts:                    *)
    Idx of int				(* Ft ::= k                   *)
  | Exp of Exp				(*     | U                    *)
  | Axp of Exp				(*     | U                    *)
  | Block of Block			(*     | _x                   *)
  | Undef				(*     | _                    *)

  and Dec =				(* Declarations:              *)
    Dec of string option * Exp		(* D ::= x:V                  *)
  | BDec of string option * (cid * Sub)	(*     | v:l[s]               *)
  | ADec of string option * int	        (*     | v[^-d]               *)
  | NDec  

  and Block =				(* Blocks:                    *)
    Bidx of int				(* b ::= v                    *)
  | LVar of Block option ref * Sub * (cid * Sub)
                                        (*     | L(l[^k],t)           *)
  | Inst of Exp list                    (*     | U1, ..., Un          *)

  and ConDec =			        (* Constant declaration       *)
    ConDec of string * mid option * int * Status
                                        (* a : K : kind  or           *)
              * Exp * Uni	        (* c : A : type               *)
  | ConDef of string * mid option * int	(* a = A : K : kind  or       *)
              * Exp * Exp * Uni		(* d = M : A : type           *)
              * Ancestor                (* Ancestor info for d or a   *)
  | AbbrevDef of string * mid option * int
                                        (* a = A : K : kind  or       *)
              * Exp * Exp * Uni		(* d = M : A : type           *)
  | BlockDec of string * mid option     (* %block l : SOME G1 PI G2   *)
              * Dec Ctx * Dec list
  | SkoDec of string * mid option * int	(* sa: K : kind  or           *)
              * Exp * Uni	        (* sc: A : type               *)

\end{verbatim} 

%% \section{Using Twelf and User Code from the SML Top Level Loop}

This section describes using Twelf and user extensions from the SML/NJ top level
loop.


\subsection{Loading Files}
First, start sml:


\begin{verbatim} 
~/save/projects/twelf/my-twelf
\$ sml
Standard ML of New Jersey v110.59 [built: Wed Sep 20 23:04:52 2006]
- 
\end{verbatim} 

Next, cd to the twelf directory, and load Twelf via
the compilation manager using the command [[CM.make]].

\begin{verbatim} 
- CM.make "sources.cm"; 
[autoloading]
[library \$smlnj/cm/cm.cm is stable]
... lots more loading
[loading (sources.cm):src/frontend/(sources.cm):frontend.sml]
[New bindings added.]
val it = true : bool
\end{verbatim} 

Next, load the user extensions, again using
the compilation manager.

\begin{verbatim} 
- CM.make "../prover/sources.cm"; 
[scanning ../prover/sources.cm]
[loading ../prover/(sources.cm):../../sml/std-lib/lib.sig.sml]
[loading ../prover/(sources.cm):../../sml/std-lib/lib.sml]
[loading ../prover/(sources.cm):canonical_lf.sml]
[loading ../prover/(sources.cm):translate.sml]
[New bindings added.]
val it = true : bool
- 
\end{verbatim} 

Finally load whatever elf file you want to manipulate,
via [[Twelf.make]]

\begin{verbatim} 
- Twelf.make ("../sources.cfg");
[Opening file ../sources.cfg]
[Closing file ../sources.cfg]
[Opening file ../prop.elf]
prop : type.
top : prop.
bot : prop.
and : prop -> prop -> prop.
/\ : prop -> prop -> prop = [x:prop] [y:prop] and x y.
imp : prop -> prop -> prop.
... lots more signature
[Closing file ../prop.elf]
val it = OK : Twelf.Status
\end{verbatim} 

Now you should be able to use the modules exported
by the [[sources.cm]] file in the Twelf directory, 
(those between [[Library]] modules [[is]]),
along with whatever code you write.  In our case
the various translators for parsing and printing.

\subsection{Using Twelf}

Some common things you'll want to do:

\begin{itemize} 
\item Print a signature.

\begin{verbatim} 
- Twelf.Print.sgn();
prop : type.
top : prop.
...
\end{verbatim} 

\item Print a signature with your own printer.

\begin{verbatim} 
- Twelf.Print.Coq.sgn();
Definition and := (fun x => (fun y => (and x y))).
Definition or := (fun x => (fun y => (or x y))).
Definition imp := (fun x => (fun y => (imp x y))).
....
\end{verbatim} 

\item Get the size of the signature

\begin{verbatim} 
- IntSyn.sgnSize();
val it = (74,0) : IntSyn.cid * IntSyn.mid
\end{verbatim} 

\item Get the abstract syntax of an element (by number)
\begin{verbatim} 
- IntSyn.sgnLookup 5;
val it = ConDec ("or",NONE,0,Normal,Pi ((#,#),Pi #),Type) : IntSyn.ConDec
\end{verbatim} 

\item Get the whole signature
\begin{verbatim} 
- map IntSyn.sgnLookup (Lib.upto(0,73));
val it =
  [ConDec ("prop",NONE,0,Normal,Uni Type,Kind),
   ConDec ("top",NONE,0,Normal,Root (#,#),Type),
   ConDec ("bot",NONE,0,Normal,Root (#,#),Type),
   ConDec ("and",NONE,0,Normal,Pi (#,#),Type),
   ...
   ConDec ("iff",NONE,0,Normal,Pi (#,#),Type),...] : IntSyn.ConDec list
\end{verbatim} 

\end{itemize} 

Now that the sandbox is complete with toys, play away!


%-------------------------------------------------------------------------------
% Bibliography                                                                  
%-------------------------------------------------------------------------------

\bibliographystyle{abbrv}
\bibliography{all}

%-------------------------------------------------------------------------------
% End                                                                           
%-------------------------------------------------------------------------------

\end{document}
