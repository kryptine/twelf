\documentclass{llncs}

\usepackage{amssymb}
\usepackage{amsmath}
\usepackage{amstext}
\usepackage{latexsym}
\usepackage{array}
\usepackage{proof}
%\usepackage{cdsty}
\usepackage{fancyheadings}
% \usepackage{pstricks}
\usepackage{pstricks,pst-node,pst-tree}
\usepackage{graphics}
%\usepackage[dvips]{graphicx}
\usepackage{graphicx}
\usepackage{psfrag}
\usepackage{code}
\usepackage{color}

\newcommand{\mygray}[1]{{\color{green}#1}}
% \newcommand{\mygreen}{\color{green}}
% \newcommand{\mygray}[1]{\em{ #1}}

\newcommand{\bangforbindingcolon}{\mathcode`!="003A}
\bangforbindingcolon
\def\sig{\mathsf{sig}}
\def\ctx{\mathsf{ctx}}
\def\kind{\mathsf{kind}}
\def\typeb{\mathsf{type}}
\newcommand{\figfoot}{\vspace{1ex}\hrule}
\newcommand{\fighead}{\hrule\vspace{1.5ex}}

\newcommand{\z}{\mbox{}}

%\newcommand{\typeLF}{\textsf{type}}
%\newcommand{\propLF}{\textsf{prop}}

%\newcommand{\false}{\textsf{false}}
%\newcommand{\true}{\textsf{true}}
%\newcommand{\andLF}{\; \textsf{and}\;}
%\newcommand{\impLF}{\;\textsf{imp}\;}
%\newcommand{\forallLF}{\;\textsf{forall}\;}
%\newcommand{\existsLF}{\;\textsf{exists}\;}
%\newcommand{\eqLF}{\;\textsf{eq}\;}
%\newcommand{\eqilLF}{\;\textsf{eqi1}\;}
%\newcommand{\eqirLF}{\;\textsf{eqi2}\;}
%\newcommand{\eqalLF}{\;\textsf{eqa1}\;}
%\newcommand{\eqarLF}{\;\textsf{eqa2}\;}

% spine notation
% \newcommand{\nil}{\textsc{nil}}
\newcommand{\comb}{\cdot}


\newcommand{\pfLF}{{\tt{prov}}}
\newcommand{\typeLF}{\tt{type}}
\newcommand{\propLF}{\tt{prop}}

\newcommand{\false}{\tt{false}}
\newcommand{\true}{\tt{true}}
\newcommand{\andLF}{\tt{and}\;}
\newcommand{\impLF}{\tt{imp}\;}
\newcommand{\forallLF}{\tt{forall}\;}
\newcommand{\existsLF}{\tt{exists}\;}
\newcommand{\eqLF}{\tt{eq}\;}
\newcommand{\eqilLF}{\tt{e1}}
\newcommand{\eqirLF}{\tt{e2}}
\newcommand{\eqalLF}{\tt{e3}}
\newcommand{\eqarLF}{\tt{e4}}

\newcommand{\orl}{\vee}
\newcommand{\andl}{\wedge}
\newcommand{\impl}{\supset}
\newcommand{\ldot}{.\,}
\newcommand{\unif}{\;\doteq\;}

%\newcommand{\andI}{\textsf{andI}}
%\newcommand{\andEL}{\textsf{andE}$_1$}
%\newcommand{\andER}{\textsf{andE}$_2$}
\newcommand{\impI}{\textsf{impI}}
\newcommand{\allI}{\textsf{allI}}
\newcommand{\allE}{\textsf{allE}}
%\newcommand{\orIR}{\textsf{orI}$_1$}
%\newcommand{\orIL}{\textsf{orI}$_2$}
%\newcommand{\orE}{\textsf{orE}}
%\newcommand{\exI}{\textsf{exI}}
%\newcommand{\exE}{\textsf{exE}}
\newcommand{\impE}{\textsf{impE}}
\newcommand{\ax}{\textsf{axiom}}
%\newcommand{\trueI}{\textsf{trueI}}
%\newcommand{\falseE}{\textsf{falseE}}

\newcommand{\listd}{\mathsf{list }}
\newcommand{\chars}{\mathsf{char}\;}
\newcommand{\integer}{\mathsf{int}\;}
\newcommand{\nil}{\mathsf{nil}}
\newcommand{\conc}{\;;\;}
\newcommand{\cons}{\mathsf{cons }\;}

\newcommand{\vd}{\vdash}
\newcommand{\vdN}{\Vdash}
\newcommand{\arrow}{\rightarrow}
\newcommand{\hastype}{\mathrel{:}}
\newcommand{\oftp}{\mathord{:}}
\newcommand{\ofvd}{\mathord{::}}
\newcommand{\lam}{\lambda}
\newcommand{\turn}{\mathord{\scriptstyle \vdash}}

\newcommand{\ednote}[1]{\footnote{\it #1}}
% \newenvironment{note}{\begin{quote}\message{note!}\it}{\end{quote}}

\newcommand{\lbb}{{[\![}}
\newcommand{\rbb}{{]\!]}}
\newcommand{\Mu}{\lbb M/u\rbb}
\newcommand{\id}{\mathsf{id}}
\newcommand{\msub}[1]{\lbb #1 \rbb}
\newcommand{\inv}[1]{{#1}^{-1}\,}

\newcommand{\type}{\mathsf{type}}
\newcommand{\mctx}{\;\mathsf{mctx}}

\newcommand{\bnfas}{\mathrel{::=}}
\newcommand{\bnfalt}{\mathrel{|}}

\addtolength{\intextsep}{-5mm}
\addtolength{\textfloatsep}{-10mm}

% Proof witnesses using higher-order logic programming
\title{Small proof witnesses for LF}
\author{
Susmit Sarkar\inst{1}\thanks{This work was supported by NSF ITR Grant 0121633:ITR/SY+SI:''Language Technology for Trustless Software Dissemination''}
\and Brigitte Pientka\inst{2}
%\thanks{This work has been partially supported by NSERC
%Grant 206263 and FQRNT Grant 206473.}
\and Karl Crary\inst{1}}%
                                                                                
\institute{%
Carnegie Mellon University, Pittsburgh, USA
\and McGill University, Montr\'eal, Canada
}
\date{}
\pagestyle{plain}
\begin{document}
\maketitle 
\begin{abstract}
We instrument a higher-order logic programming search procedure to
generate and check small proof witnesses for the Twelf system, an
implementation of the logical framework LF. In particular, we extend
and generalize ideas from Necula and Rahul~\cite{Necula+01:oracle} in
two main ways: 1) We consider the full fragment of LF including
dependent types and higher-order terms and 2) We incorporate caching
of sub-proofs to generate even more compact proof representations. Our
experimental results demonstrate that many of the restrictions in
previous work can be overcome and generating and checking small
witnesses within Twelf provides valuable addition to its general
safety infrastructure.
\end{abstract}

\section{Introduction}
Proof-carrying code applications establish trust by verifying
compliance of the code with safety and security policies.
A code producer verifies that the program is safe to
run according to some predetermined safety policy, and supplies a
binary executable together with its safety proof. Before
executing the program, the code consumer then quickly checks the code's
safety proof against the binary. 

The Twelf system \cite{Pfenning99cade}, an implementation of the logical framework LF \cite{Harper93jacm}, provides a general safety
infrastructure to represent and execute safety policies via a
higher-order logic program interpretation and has been employed in
several proof-carrying code projects
\cite{AppelFelty00,Crary:POPL03,AppelFelten99,Crary:CADE03}.   
Higher-order logic programming extends first order logic
programming along two orthogonal dimensions: First, dynamic assumptions
may be generated and used during proof search. Second, first-order terms
are replaced with dependently typed $\lambda$-terms, thereby
directly supporting encodings via higher-order abstract
syntax. 

One of the benefits of using Twelf is that the
execution of a query will not only produce a yes or no answer, but
produce a proof term as a certificate that can be checked independently. 
This increases the confidence in the overall correctness of the
higher-order logic programming engine, and the certificate can
be sent to the code consumer where compliance with the code is
checked. 
%These features make higher-order logic
%programming an ideal generic framework for implementing and
%experimenting with different safety policies and reduces the effort
%required for each particular safety policy.

In this paper, we show how to instrument the higher-order logic
programming interpreter to generate and check small proof
witness by extending and generalizing ideas by Rahul and
Necula\cite{Necula+01:oracle}. To obtain small proof witnesses, they
propose to only record the non-deterministic choices during 
logic programming execution as a bit-string. We can check such a proof
witness by guiding a deterministic logic programming interpreter using the
bit-string and re-running the proof. This simple idea has
been proven to be effective in many practical examples and has also
been used by Wu {\em{et. al}} \cite{Appel:PPDP03} for creating a
foundational proof checker with small witnesses. However, all these
approaches restrict themselves to a fragment of LF excluding
higher-order terms and dependent types thereby trading the expressive
power of the logical framework LF against simplicity of implementation
to generate and check proof witnesses.  
As a consequence, these systems do not support higher-order abstract
syntax in practice, but each particular system now has to use
encoding tricks to encode their variable binding constructs together
with substitution operations. For example, Wu {\em et
  al}\cite{Appel:PPDP03} encode the explicit  substitution calculus
\cite{Abadi:POPL90} together with the necessary proofs about
substitutions for their foundational certified code implementation. As
the technology of certified code evolves, we will move to more
powerful and expressive safety policies and type systems and the use
of higher-order abstract syntax will become crucial for achieving a 
simple, compact encoding of these systems.


In this paper, we describe the design of generating and checking
of small proof witnesses for the full logical framework LF. This work
continues where Necula and Rahul \cite{Necula+01:oracle} left off 
saying ``more experimental results are needed especially in the
higher-order setting''. Our work has been implemented and evaluated
within the Twelf system \cite{Pfenning99cade} making it unnecessary to
build separate proof checking engines. To obtain a practical scalable
implementation, we use higher-order substitution tree  indexing
\cite{Pientka:ICLP03}. Furthermore, we improve on the size of proof
witnesses by caching common sub-proofs\footnote{Eliminating
  common sub-proofs is an orthogonal problem to eliminating redundant
  implicit type information, as is proposed in \cite{Necula98lics}.}
. 
%Identifying and factoring out common
%sub-proofs leads to more compact oracles and can decrease
%proof-checking time by a factor of ?? since common sub-proofs are only
%checked once.  

%   However, all these approaches are
% restricted to simply typed Prolog-like engines which work on the
% fragment of hereditary Harrop formulas. Although theses approaches
% allow dynamic assumptions, they disallow a a higher-order term
% language where terms can be defined using $\lambda$-abstraction. This
% means they can only support first-order abstract syntax rather than
% higher-order abstract syntax.  The main reason for staying with a
% simple first-order term language are that unification is decidable and
% easily implemented. Moreover, 

%two-folded: First, in simple safety policies about typed-assembly
%language, higher-order abstract syntax is rarely used. However, as we
%move to more complex safety policies, support for higher-order
%abstract syntax will become more desirable. Second, techniques needed
%to extend oracle-based proof compression and proof re-construction for
% higher-order terms have been 

%% However, the lack of support for higher-order abstract syntax
%% encodings, means that any variable binding constructs must be
%% explicitly encoded. The overhead in their setting is still
%% manageable although annoying, since simple safety policies about
%% typed assembly require variable binding constructs only in few
%% places. However, as we move beyond proving memory safety, we predict
%% that richer safety policies and type systems will play a more
%% important role in proof-carrying code applications. For these richer
%% language, support for higher-order abstract syntax will be
%% crucial. 


This paper is structured as follows. We give background on
higher-order logic programming in Twelf in Section~\ref{sec:twelf}. In
Section~\ref{sec:oracles}, we present our approach to generating and
checking small proof witnesses. In Section~\ref{sec:indexing} we explain 
higher-order term indexing and in Section~\ref{sec:tabling}, we
discuss caching techniques for factoring out common subproofs. We
conclude with a discussion of some experimental results within Twelf
and  related work.

\section{Higher-order logic programming}\label{sec:twelf}

%% Higher-order logic programming in Twelf extends first-order logic
%% programming in three main ways: First, first-order terms are replaced
%% with (dependently) typed $\lambda$-terms. Second, the body of clauses
%% may contain implications and universal quantification, thereby
%% generating dynamic assumptions which may be used during proof
%% search. Thirdly, execution of a query will produce more than a yes or
%% no answer, and generate a proof term as a certificate which can be
%% checked independently. These features make higher-order logic
%% programming an ideal generic framework for implementing formal safety
%% policies given via axioms and inference rules and executing them.

The theoretical foundation underlying higher-order logic programming
within Twelf is the LF type theory, a dependently
typed lambda calculus  \cite{Pfenning91lf}. In this setting types are interpreted as
clauses and goals and typing context represents the store of program
clauses available. We will use types and formulas
interchangeably. Types and programs are defined as follows: 

\begin{minipage}[b]{6cm}
\[
\begin{array}{lcl}
\mbox{Types } A & ::= & P \mid  A_1 \rightarrow A_2 \mid \Pi x:A_1.A_2 \\
\mbox{Terms }  M & ::= & c \comb S \mid x \comb S \mid \lambda x. M  
\end{array}
\]
\end{minipage}
\begin{minipage}[b]{6cm}
\[
\begin{array}{lcl}
\mbox{Programs }  \Gamma & ::= & \cdot \mid \Gamma, x:A \\
\mbox{Spines } S & ::= & \nil \mid M ; S
\end{array}
\]
\end{minipage}

$P$ ranges over atomic formulas such as $a \cdot S$, where $a$ is a
type constant. We interpret the function arrow $A_1 \rightarrow A_2$
as implication and the $\Pi$-quantifier, denoting dependent function
type, corresponds to the universal $\forall$-quantifier. Types, which
are goals and clauses, are inhabited by corresponding proof terms $M$,
and we assume that all proof terms are in normal form. 

Other higher-order logic programming languages of a similar
flavor are $\lambda$-Prolog \cite{Nadathur99cade} or
Isabelle\cite{Paulson86}. To illustrate the notation and explain the
problem of small proof witnesses, we will first give an example of
encoding the natural deduction calculus in the logical framework LF
using higher-order logic programming following the methodology in
\cite{Harper93jacm}. For more information on how to encode formal
systems in LF see \cite{Pfenning97}.  Using this example, we will
explain generating and checking of small proof witnesses.

\subsection{Representing Logics}
As a running example, we will consider a fragment of intuitionistic natural
deduction calculus consisting of implications and universal quantifiers. Propositions can
be then described as follows:

\[
\begin{array}{llll}
\mbox{Propositions} & A,B, C & := & \ldots \mid A \impl B \mid \forall x.A \\
\mbox{Context} & \Gamma & := & \ldot \mid \Gamma,  A
\end{array}
\]

Inference rules describing natural deduction are presented next.

\[
\infer[{\textsf{allI}}]{\Gamma\vdash \forall x. A}
{\Gamma\vdash [a/x]A & a \mbox{ is new}}
\qquad
\infer[{\textsf{allE}}]{\Gamma\vdash [T/x]A}
{\Gamma\vdash \forall x.A}
\qquad
\infer[{\textsf{hyp}}]{\Gamma, A \vdash A}
{}
\]
\[
\infer[{\textsf{impI}}]{\Gamma\vdash A\impl B}
{\Gamma,A\vdash B}
\qquad
\infer[{\textsf{impE}}]{\Gamma\vdash B}
{\Gamma\vdash A\impl B
\quad
\Gamma\vdash A}
\]

To represent this system in LF, we first need formation rules to
construct terms for propositions.  We intend that terms belonging to
{\tt prop} represent well-formed propositions and {\tt i} represents individuals.  
%prop : type.
%i    : type.
%\vspace{0.1in}
%false  : prop.
%true   : prop.

%\noindent
%\begin{code}
%imp : prop -> prop -> prop.       forall : (i -> prop) -> prop.
%\end{code}
%
The connective for implication has type takes in two propositions and
returns a proposition, hence the constructor {\tt imp} has type {\tt prop -> prop -> prop.} To represent the forall-quantifier, we will use
higher-order abstract syntax. The crucial idea is to represent bound
variables in the object language (logic) with bound variables in the
meta-language (higher-order logic programming). Hence the type of {\tt
  forall} is {\tt (i -> prop) -> prop}.

Next we turn our attention to the inference rules. The 
judgment for provability within this logic is denoted by the
type family {\tt prov}.
%
%\begin{code}
% prov : prop -> type.
%\end{code}%
%
Each clause will correspond to an inference rule in the object
logic. For convenience, we give the constructors descriptive names.
% ,and follow the order of the inference rules presented earlier.

\hspace{-0.65cm}
\begin{small}
\begin{minipage}[t]{5.5cm}
\begin{code}
alli: prov (forall $\lambda$x. A x)
      <- $\Pi$x. prov (A x)
alle: prov (A T)
      <- prov (forall $\lambda$x. A x).
 \end{code}
 \end{minipage}
\begin{minipage}[t]{5.5cm}
\begin{code}
impi: prov (imp A B)
      <- (prov A -> prov B).
impe: prov B
      <- prov (imp A B)
      <- prov A.
\end{code}
\end{minipage}
  
\end{small}

%\z
%truei    : prov true.
%\z
%falsee   : prov C
%            <- prov false.

$A$, $B$, $C$ denote existential or logic variables which are
instantiated during proof search. Throughout the example we reverse
the arrow {\tt{A -> B}} writing instead {\tt{B <- A}}. This way, goals
appear in the order in which they are processed during proof
search. From a logic programming view, it might be more intuitive to
think of the clause {\tt{H <- A$_1$ <- A$_2$ <- $\ldots$ <- A$_n$}} as
{\tt{H <- A$_1$, A$_2$, $\ldots$, A$_n$}}. There are two key ideas
which make the encoding of the logic calculus elegant and direct.
First, we use and manipulate dynamic assumptions which higher-order
logic programming provides, to eliminate the need to manage
assumptions in a list explicitly. To illustrate, we consider the
clause {\tt impI}. To prove {\tt prov (imp A B)}, we prove {\tt prov
B} assuming {\tt prov A} In other words, the proof for {\tt prov B}
may use the dynamic assumption {\tt prov A}.  Second, we use
higher-order abstract syntax to encode the bound variables in the
universal quantifier. As a consequence substitution in the object
language can be reduced to application and $\beta$-reduction in the
meta-language (higher-order logic programming). Consider the rule for
all-elimination. If we have a proof of $\forall x.A$ , then we know
that $[T/x]A$ is true for any term $T$. The substitution $[T/x]A$ in
the object language is achieved via application in the meta-language
{\tt (A T)}.


\subsection{Proof search in higher-order logic programming}

Higher-order logic programming is similar to a Prolog interpreter in
that it performs essentially a depth-first search over all the program
clauses. The key challenges in moving to a higher order setting are
twofold: First, we may have dynamic assumptions which may be
used within a certain scope. Second, since we allow higher-order
terms (i.e. terms may contain $\lambda$-abstraction), higher-order
unification is used to unify clause heads with current goal. 
%Third,
%proof terms are generated which represent the proof the interpreter
%found. These features make a language like Twelf ideal for certified
%code systems.

In this section,  we briefly describe the depth-first proof search
procedure of the higher-order logic programming
interpreter. Computation in logic programming is achieved through
proof search. Given a goal (or query) $G$ and a program $\Gamma$, we
derive $G$ by successive application of clauses of the program
$\Gamma$. 
%Following Miller {\em{et al}} \cite{Miller91apal}, we
%interpret the connectives in a goal $G$ as {\em{search instructions}}
%and the clauses in $\Gamma$ as specifications of how to continue
%search when the goal is atomic. 
To solve a goal $G$ from a set of clauses $\Gamma$, we decompose the
compound goal $G$ until it is atomic and then resolved it with a
program clause. We have the following three possible actions (for a
more detailed description see \cite{Miller91apal}):

%A proof is said to be
%{\em{goal-oriented}} if every compound goal is immediately decomposed
%and the program is accessed only after the goal has been reduced to an
%atomic formula. A proof is {\em{focused}} if every time a program
%formula is considered, it is processed up to the atoms it defines
%without need to access any other program formula. A proof having both
%these properties is {\em{uniform}} and a formalism such that every
%provable goal has a uniform proof is called an abstract logic
%programming language.  
%In the subsequent description, we will concentrate on the
%goal-oriented step. To solve a goal $G$ from a set of clauses $\Gamma$
%(written as $\Gamma \vd G$), we have the following three possible
%actions:   

%\begin{figure}[h]
%\fighead
% \begin{center}
\begin{small}
% \noindent \mbox{{\bf{Solve Goal $\Gamma \vd M: G$:}}\hfill}% \\[-2.5em]
\begin{description}
\item[Select] $\Gamma \vd  G \Rightarrow c_i \cdot S$ \\
    \mbox{Given an atomic goal $G$ and clauses $\Gamma$:}\hfill\\
     Focus on a clauses $c_i : A_i$ from $\Gamma$ to establish a proof
     $c\cdot S$ for $G$, by unifying the head of $A_i$ with the current
     goal $G$. 

\item[Augment] $\Gamma \vd  G_1 \arrow G_2 \Rightarrow \lambda u. M$ if $\Gamma,
  u\oftp G_1 \vd M : G_2 \Rightarrow W$ \\
Augment the clauses in $\Gamma$ with the dynamic assumption $u{:} G_1$ and
establish a proof $M$ for the goal $G_2$ from the extended program
$\Gamma, u \oftp G_1$. 
\item[Universal] $\Gamma \vd  \Pi x. G \Rightarrow \lambda x. M$ if $\Gamma \vd
  [a/x]G\Rightarrow [a/x]M$ where $a$ is a new parameter\\
Given a universally quantified goal $\Pi x. G$, we generate a new parameter $a$, and establish a $[a/x]M$ proof  for $[a/x]G$ in the program context $\Gamma$.
\end{description}
%   \caption{Solve goal $G$ from clauses in $\Gamma$}
%   \label{fig:solve}
\end{small}    
% \end{center}

%\figfoot
%\caption{\label{fig:solve}Solve goal $G$ from clauses in $\Gamma$}
%\end{figure}

Once the goal is atomic, we need to select a clause from the
program context $\Gamma$ to establish a proof for $G$. In a logic
programming interpreter, we consider all the clauses in $\Gamma$ in order. 
First, we will consider the dynamic assumptions, and then we will try
the static program clauses one after the other. 
Let us assume, we picked a clause $A$ from the program context
$\Gamma$ and we now need to establish a proof for $G$, by unifying the
head of the clause $A$ with $G$ and solving the subgoals of $A$.
%Note that during proof search we typically have the program
%clauses $\Gamma$ and the goal $G$ we are trying to prove from the
%clauses in $\Gamma$ as inputs, while the proof term $M$ is the output
%of the search. 
We will illustrate proof search by considering the following example:  

\begin{code}
prov (forall $\lambda$y. (imp (forall $\lambda$ x. p x) (p y)))  
\end{code}

which corresponds to $(\forall y. (\forall x.p(x)) \impl p(y)$).  
where {\tt p} is a defined predicate. To prove the query, we will
start by unifying the head of the clause ({\tt allI}) with the
query, which results in subgoal:  

\[
\begin{array}{c}
\Pi a. \pfLF ( \impLF (\forallLF \lambda y. p\; y)\; (p a))
\end{array}
\]

In the {\sf{Universal}} step, we introduce a new parameter $a$
yielding the subgoal:
\[
\begin{array}{c}
\pfLF ( \impLF (\forallLF \lambda y. p\; y)\; (p a)).
\end{array}
\]

To prove this subgoal will again inspect our clauses. Three of them
will be applicable, namely {\tt allE}, {\tt impI}, and {\tt
  impE}. This time we will pick the second clause {\tt impI}. Hence we
will introduce the dynamic assumption ${\tt{u}} {:} \pfLF (\forallLF
\lambda \;y. p\; y)$ and show $\pfLF\; {\tt (p\; y)}$ using the dynamic
assumption {\tt{u}}. In the third step, again two clauses are
applicable,  {\tt allE}, and {\tt impE}. Using the first one, {\tt
  allE}, we need to show that we can prove $\pfLF (\forallLF \lambda
y. P\; y)$. There are four possible clauses whose clause head will
unify: the dynamic clause {\tt u} and the three program clauses {\tt
  alli}, {\tt alle}, and {\tt impe}. Using the dynamic assumption {\tt
  u}, we can finish the proof. Twelf's
higher-order logic programming engine will generate the following
proof term in explicit form:  

% \hspace{-1.25cm}
%\begin{minipage}[h]{12cm}
\begin{code}
(alli {\mygray{($\lambda\!\!$ x. ((forall $\lambda\!\!$ y. p y) imp p x))}}
   $\lambda\!\!$ a. (impi {\mygray{(forall $\lambda\!\!$ y.p y) (p a)}}
           $\lambda\!\!$ u. (alle {\mygray{($\lambda\!\!$ y.p y)}} a u))).
\end{code}
%\end{minipage}

The final proof term not only tracks the rules which have been used in
every step of the proof, but also tracks the instantiations for the logic
variables in each steps. In the proof term above we show the
instantiations in gray.


% From an operational view point, the search can be described as
% follows: 

%\begin{table}[htbp]
%\fighead
%\begin{small}
%Solve Goal $\Gamma \vd M : G$:
%\begin{enumerate}
%\item $\Gamma \vd (c \cdot S) : G$ \\
%    \mbox{Given an atomic goal $G$ and clauses $\Gamma$:}\hfill\\
%     Focus on a clauses $c : A$ from $\Gamma$ to establish a proof $S$ for $G$.

%\item $\Gamma \vd \lambda u.M : G_1 \arrow G_2$ if $\Gamma, u\oftp G_1
%  \vd M : G_2$ \\
%Augment the clauses in $\Gamma$ with the dynamic assumption $u : G_1$ and
%establish a proof $M$for the goal $G_2$ from the extended  program $\Gamma, u \oftp G_1$.
%\item $\Gamma \vd \lambda a.M : \Pi x. G$ if $\Gamma \vd M : [a/x]G$ where $a$ is a new parameter\\
%Given a universally quantified goal $\Pi x. G$, we
%generate a new parameter $c$, and establish a proof $M$ for $[c/x]G$ in the
% program context $\Gamma$.
%\end{enumerate}
%\end{small}
%\figfoot 
%\caption{Solve goal $G$ from clauses in $\Gamma$}
%\label{tab:solve}
%\end{table}

%Once the goal is atomic, we need to select a clause from the
%program context $\Gamma$ to establish a proof for $G$. In a logic
%program interpreter, we consider all the clauses in $\Gamma$ in order. 
%First, we will consider the dynamic assumptions, and then we will try
%the static program clauses one after the other. 
%Let us assume, we picked a clause $A$ from the program context
%$\Gamma$ and we now need to establish a proof for $G$.

%\begin{table}[h]
%\fighead
%\begin{small}
%Focus on clause $ c : A$ to solve atomic goal $P$.
%\begin{enumerate}
%\item $\Gamma > P' \vd \nil : P$ \\
% Given the atomic clause $P'$ with name $n$, we establish a proof for the
%  atomic goal $P$, by checking  $P' = P$. If yes then
%  succeed. Otherwise fail and backtrack.  
%\item $\Gamma > n : G_2 \arrow G_1 \vd (M ; S) : P$ 
%      if $\Gamma > G_1 \vd S : P$ and  $\Gamma \vd M : G_2$ \\
%  Given the clause $G_2 \arrow G_1$ with name $n$, we establish a
%  proof of the atomic goal $P$, by trying to use the clause $G_1$ to
%  establish a proof $M$ for $P$. If it succeeds, we establish a proof
%  $S$ for the goal $G_2$. If it fails,  backtrack. 
%\item $\Gamma > n : \Pi x\oftp A_1. A_2 \vd S : P$ if $\Gamma > n :
%  [T/x]A_2 \vd (T ; S) : P$\\
%  Given the clause $\Pi x\oftp A_1. A_2$ with name $n$, we
%  establish a proof $(T ; S)$for the atomic goal $P$ by instantiating
%  $x$ with a term $T$, and use the clause $[T/x]A_2$
%  to establish a proof $S$ for the atomic goal $P$. 
%\end{enumerate}
%\end{small}
%\figfoot
%  \caption{Focus on clause $c:A$ to solve atomic goal $P$}
%  \label{tab:foc}
%\end{table}

As shown in Necula~\cite{Necula98lics}, the instantiations of
existential variables need not be recorded in the explicit proof terms
but can be reconstructed as long as we only concentrate on a fragment
of LF, called LF$_i$. This can lead to substantial savings in proof
checking and proof size. Proofs are roughly $\mathrm{O}(\sqrt{n})$,
where $n$ is the size of the query. However, extending this idea to
full LF has been difficult \cite{Reed04lfm}. Maybe more importantly,
proofs in LF$_i$ are still several times as big as the overall program
they certify.

Our goal is to produce smaller proof witnesses by reducing the proof
evidence to the choices we make while constructing the proof.  In the
previous example, it suffices to know that in the first step, three
possible rules apply, namely {\tt alli}, {\tt alle}, and {\tt impe}
and we want to follow the first possibility. In the second step, again
three possible rules apply, namely {\tt alle}, {\tt impi}, and {\tt
impe}, and we want to follow the second possibility. In the final
step, we have four potential candidates, the dynamic assumption ${\tt
u}{:} \forallLF (\lambda y. p\; y)$, and the rules {\tt allI}, {\tt
allE}, and {\tt impE}.  Hence it would suffice to store only a list of
the choices made in the proof. In this example, the choices can be
characterized by the following sequence: $1/3$, $2/3$, $1/2$, $1/4$,
keeping in mind that dynamic assumptions are tried first by proof
search procedures. This sequence will constitute our compact proof
witness and is all that needs to be generated and sent to the
verifier. In the remainder of the paper, we show how to incorporate
this technique into Twelf.

\section{Generating and checking small proof witnesses}
\label{sec:oracles}

%In this section, we describe the oracle-based proof generation and
%checking. To generate oracles we can follow two ways. One way is to
%modify the logic-programming proof search procedure to generate a
%bit-string during proof search directly. As logic programming's
%depth-first search is incomplete in general, this may not work in all
%cases, however this may be viable with more sophisticated theorem proving
%technology. In certifying code systems in particular, the safety proof
%(i.e. proof term) can be generated by a certifying compiler, and
%we merely aim at compressing the safety proof into a bit-string. This
%enables us to use a second strategy, where we pass into the search procedure
%described above a proof term to guide the proof search. This way, we
%can eliminate all non-determinism from the previous search procedure.

\subsection{Proof compression}
In this section, we describe the modifications needed to generate
a compact proof witness in form of a bit-string using the verifier's
proof search procedure from the previous section. 
We assume that we already have the full proof term and we are merely
interested in compressing the proof term to a small witness in form of
a bit-string. This is not a restriction, since in certifying code
systems the safety proof are typically generated by a compiler. 
%Of course, we can also try to generate the with more sophisticated 
%(i.e. proof term) can be generated by a certifying compiler, 
The bit-string encodes the non-deterministic choices within the proof,
namely picking the right clause $c_1{:}A$ from the program context
$\Gamma$ to establish a proof $S$ for $P$, once the goal $P$ is
atomic, by unifying the head of $A$ with the atomic goal
$P$. Potentially, there is more than one clause whose head unifies
with $P$, and hence a proof search procedure would need to try all the
possible choices in order. The proof witness just needs to keep track
of what possibility was successful.

%% This leads to an efficient encoding of the non-deterministic
%% choices. If we have $k$ possibilities, we need $\lceil\log_2 k\rceil$
%% bits. Since we know that the path through the proof tree always has to
%% lead to a proof, if there is only one choice applying, this formula
%% correctly says that we do not need to emit an advice.  The verifier
%% knows how many choices apply, so it can calculate the number of bits
%% to pick off the oracle. This means that we do not need an explicit
%% separator between the choices, although we include them here for
%% better readability. The witness corresponding to the non-deterministic
%% choices in the previous example is: {\tt 100010101000}.
%%
%% It turns out we usually count in unary! -- Susmit
%% I don't understand your comment. In any case you must talk about
%% the proof witness and how it will look like in binary form - bp.

For the idea to work, generating and checking witnesses have to
perform  the same overall proof search. Moreover, the order 
in which different choices are considered must be the same.  The only
difference is that in proof search we would explore possibly multiple
fruitless paths, backtracking until we find the right path. When
generating and checking witnesses, we will consult the proof term
or witness respectively to know which choice to consider, and thus
eliminate uses of backtracking.
We modify the proof search steps presented earlier in the following
way:
%that we pass
%in the proof term together with the goal and output a proof witness in
%form of a bit-string. 
% Instead of building up a proof term recursively
% in the {\sf{Select}}, {\sf{Augment}} and {\sf{Universal}} step, we
% only generate the proof witness in the modified {\sf{Select}} step.

%\begin{figure}[h]
%\fighead
%\begin{center}
%\parbox[h]{12cm}{
%Witness generation:\hfill}
\begin{small}
% \noindent \mbox{{\bf{Solve Goal $\Gamma \vd M: G$:}}\hfill}% \\[-2.5em]
\begin{description}
\item[Select] $\Gamma \vd c_i \cdot S: G \Rightarrow
  \underset{1 \ldots (i-1)}{\underbrace{0\ldots 0}}1
%\underset{(i+1) \ldots k}{\underbrace{0\ldots 0}}
W $ \\
    \mbox{Given an atomic goal $G$ and clauses $\Gamma$:}\hfill\\
%    Let $k$ be the number of clauses whose head unifies with the
%    current goal $G$ and 
     Let $c_i : A_i$ be the $i-$th clause which leads to a
    proof $c_i\cdot S$ of $G$ from $\Gamma$. \\
    Focus on clauses $c_i : A_i$ from $\Gamma$ to compress a proof
     $c\cdot S$ for $G$ to $W$.

\item[Augment] $\Gamma \vd   \lambda u. M : G_1 \arrow G_2 \Rightarrow
  W$ if $\Gamma,
  u\oftp G_1 \vd G_2 \Rightarrow M$ \\
Augment the clauses in $\Gamma$ with the dynamic assumption $u{:} G_1$ and
compress a proof $M$ for $G_2$ within the extended program
$\Gamma, u \oftp G_1$ to obtain the witness $W$. 
\item[Universal] $\Gamma \vd  \lambda x. M : \Pi x. G \Rightarrow W$ if $\Gamma \vd
  [a/x]M: [a/x]G\Rightarrow W$  \\ %where $a$ is a new parameter\\
Given a universally quantified goal $\Pi x. G$, we generate a new
parameter $a$, and compress a proof $[a/x]M$  for $[a/x]G$ in the
program context $\Gamma$ to $W$.\\
\end{description}
%   \caption{Solve goal $G$ from clauses in $\Gamma$}
\end{small}    
%
%\end{center}

Note that the {\sf{Select}} step is deterministic as the proof term
determines which choice will be successful. It should be intuitively
clear that, we do not necessarily have to pass in the full proof term,
but could directly produce a proof witness in form of a bit-string, if
our proof search is powerful enough that it will eventually find a proof.

%\figfoot
%\caption{\label{fig:pwgen}Proof Compression}
%\end{figure}


%The oracle-string encodes one type of possible choices, but as we
%briefly mentioned earlier, unification in the higher-order setting is
%undecidable in general. In Twelf, we use a higher-order pattern
%unification algorithm together with constraints. Higher-order pattern
%unification is in fact decidable. Although it is too restrictive to
%concentrate solely on higher-order patterns statically, 
%most of the non-patterns encountered (for example $(A\;T)$), will be a
%pattern during execution. Since we assume that we already found a proof $M$
%for a goal $G$ without any left-over unification constraints, 
%all the unification problems solved during execution were
%decidable. It is worth pointing out that our proposal differs here
%from the proposal of Necula and Rahul, who propose to encode the
%non-deterministic choice of Huet's unification algorithm, which does
%not distinguish between higher-order patterns and non-patterns. 

\subsection{Checking small proof witnesses}
In this section, we modify the previous search procedure, in such a
way that it is not parameterized by the proof term $M$, but rather by
the compact proof witness $W$ encoded as a bit-string. Therefore we
have : Solve Goal $\Gamma \vd W : G$, where $W$ is a compact proof
witness.   

%\begin{figure}[h]
%\fighead
%\begin{center}
%\parbox[h]{12cm}{
%Witness checking via proof reconstruction:}
\begin{small}
% \noindent \mbox{{\bf{Solve Goal $\Gamma \vd M: G$:}}\hfill}% \\[-2.5em]
\begin{description}
\item[Select] $\Gamma \vd \underset{1 \ldots
    (i-1)}{\underbrace{0\ldots 0}}1
%\underset{(i+1) \ldots    k}{\underbrace{0\ldots 0}}
W : G $ \\
    \mbox{Given an atomic goal $G$ and clauses $\Gamma$:}\hfill\\
    Let $k$ be the number of clauses whose head unifies with the
    current goal $G$, then inspect up to $k$ bits, and
    find the $i$-th bit which is one. \\ 
    Focus on clauses $c_i : A_i$ from $\Gamma$ to establish a proof
    for the atomic goal $G$ from $\Gamma$ using remaining proof witness $W$.
\end{description}
%   \caption{Solve goal $G$ from clauses in $\Gamma$}
\end{small}    
%}
%\end{center}
%\figfoot
%\caption{\label{fig:pwrecon} Proof Reconstruction}
%\end{figure}

In {\sf{Select}} step, we first generate the $k$ possible candidates
whose head will unify with the current goal $G$. If $k$ is greater
than 1, we will examine up to $k$ bits from the witness to see which
choice to take. If a bit $1$ occurs at position $i$ of these $k$
bits, we will pick the $i$-th candidate. For this idea to work in practice, it is crucial that the order of choices during witness checking is same as during witness generation.

In order to check the proof witnesses, we re-run the prover guided
with the advice encoded in the bit-string. The witness checker is then
in fact a deterministic search procedure. No backtracking is
necessary, since all the non-deterministic choices are resolved.  
%By
%taking the appropriate branches (and performing the needed
%unification), the user can be convinced that such a proof exists. 
Note that the proof term does not need to be reconstructed. This can
lead to savings of time and memory. On the other hand, paranoid
consumers may need more assurance than trusting our implementation
(including complex algorithms such as higher-order pattern
unification). In such cases, we have an option of regenerating the
proof term by instrumenting the search procedure above.  This amounts
to decompressing the proof witness $W$to an explicit proof term $M$
and use a different trusted type-checker to verify the expanded proof
witness.

\subsection{Bit-string encodings for proof witnesses}

The choices as described above are choice sequences of the form
$i_1/k_1,i_2/k_2,\ldots$, where at the $j^{th}$ stage we have $k_j$
choices, and we want to pick $i_j$ ($1 \leq i_j \leq k_j$). With the
tight coupling of the witness generation and checking phases, both
phases agree on the number of choices ($k_j$) as well as the ordering
of those choices. That is, both producer and checker agree on which
choice is to be considered the $i_j^{th}$ one.

We can now see that the separator between choices is unnecessary. We
can decide on a encoding scheme, and pull only the requisite number of
bits from the oracle. The witness checker will always know how many
bits to extract.

We have experimented with two simple encoding schemes, though more
complex coding schemes can be imagined. The original proposal by
Necula and Rahul proposed what we call the binary scheme, in that the
number would be encoded in binary. If $k$ choices apply, this will
require $\lceil\log k\rceil$ bits. We discover that a scheme we call
unary encoding works better in practice. In this scheme, the choice
number $i$ is encoded as $0 0 0 \ldots (\mbox{i-1 zeros}) 1$. This
takes $i$ bits.

The binary scheme will work better when we habitually have a large
number of choices, and we take one of the later choices in the
ordering considered by the producer/checker. The unary scheme will
work better precisely in the other cases. In all our examples, we have
observed that only a few choices typically apply. Further, logic
programmers usually write their programs so that the more common
choices are tried first. With these observations, unary encodings
should outperform binary encodings, as indeed they do in experimental
studies. This is a configurable option in our engine, and can be set
depending on the particular proof or logic.

\section{Optimizations}
\subsection{Higher-order term indexing}\label{sec:indexing}

A proof search procedure must have a way of retrieving all clauses of
the logic program which may satisfy the current goal, since such a
method will dictate how many choices we are returned at any step.
Most first-order logic programming interpreter use term indexing
strategies such as automata driven indexing
\cite{Ramakrishnan01:indexing} to efficiently retrieve all clauses
whose head unifies with the current goal.  However, indexing
strategies for higher-order terms are difficult, since in general
retrieval and often also insertion operations rely on computing the
most general unifier or the most specific generalization. However, in
the higher-order case, unification is in general undecidable and the
most general unifier does not necessarily exist. The same holds for
computing the most specific generalization of two terms.

%As
%discovered by Miller \cite{Miller91iclp}, there exists a decidable
%fragment, called higher-order patterns. For this fragment, unification
%and computing the most specific generalization is decidable even in
%rich type theories with dependent types and polymorphism as shown by
%Pfenning \cite{Pfenning91lics}.  However, these algorithms, which must
%compute bound variable dependencies, may not be efficient in practice
%\cite{PientkaPfenning:CADE03}.  
We will adopt higher-order substitution trees
\cite{Pientka:ICLP03,Pientka03phd} to index higher-order logic
programs. Substitution tree indexing has been successfully used in a
first-order setting \cite{Graf+Book95} and  allows the sharing of
common sub-expressions via substitutions. This is unlike other
non-adaptive term indexing, which only allow sharing of common term
prefixes. To extend substitution tree indexing to the higher-order
setting, we will concentrate on the fragment of linear higher-order
patterns\cite{PientkaPfenning:CADE03}, where all existential variables
must occur only once and are applied to all distinct bound
variables. Linear higher-order patterns refine the notion of
higher-order patterns  \cite{Miller91iclp}, where all existential
variables must be applied to a some distinct bound variables.  In
\cite{Pientka:ICLP03,Pientka03phd}, we give a formal description for
computing the most specific generalization of two linear higher-order
patterns, for inserting terms in the index and for retrieving a set of
terms from the index s.t. the query is an instance of the term in the
index, and show correctness. The construction of a substitution tree
in the higher-order setting follows the overall algorithm described in
\cite{Ramakrishnan01:indexing}. Here we will illustrate higher-order
substitution trees by an example. 

To describe higher-order substitution trees, it is crucial to
distinguish between bound variables, existential variables and
``internal'' existential variables. We will assume that any
existential variable is applied to all bound variables in whose
context it occurs in to facility simple and efficient insertion and
retrieval algorithms. A higher-order substitution tree is a node with
substitution $\rho$ whose range is $\Sigma$ and every child node has a
substitution $\rho_i$ with domain $\Sigma$. In other words, any
internal existential variable in $\Sigma$ will be defined at a later
node in a path in the substitution tree and for every path
from the top node $\rho_0$ with range $\Sigma_0$ to the leaf node
$\rho_n$ we have the range of the substitution which is the
composition of all the substitutions along this path, is empty. In
other words, there are no internal existential variables left after we
compose all the substitutions $\rho_n$ up to $\rho_0$. 
%
%For this setup to work cleanly in the higher-order
%setting, it is crucial that we distinguish between existential
%variables in $\Delta$ and bound variables and assumptions in
%$\Gamma$. Moreover, it is essential that existential variables allow
%in place up-date. In particular, we will rely on the notion of
%existential variables which are ``fully applied'', i.e. they may
%depend on all the bound variables they occur in.
To illustrate, consider the following clauses describing part of a
conversion of formulas into prenex normalform. 

\begin{small}
\[
\begin{array}{l}
%\eqLF :\;  \propLF \rightarrow \propLF \rightarrow \typeLF.\\[1em]
%
\eqilLF: \eqLF (\impLF (\existsLF \lambda x. A\; x)\; B)\quad (\forallLF \lambda x. (\impLF (A\; x)\; B)).\\
\eqirLF: \eqLF (\impLF A\; (\forallLF \lambda x. B\;x)) \quad (\forallLF \lambda x. (\impLF A \; (B\; x))).\\
%\eqalLF: \eqLF (\andLF (\forallLF \lambda x. A \;x) \; B)\quad (\forallLF \lambda x. (\andLF (A\; x)\; B)).\\
\eqalLF: \eqLF (\andLF A \; (\forallLF \lambda x. B\; x)) \quad (\forallLF \lambda x. (\andLF A \; (B\;x))).\\
\end{array}
\]
\end{small}

We see that the four given clauses share a lot of structure. For
example clause $\eqilLF$ and $\eqirLF$ ``almost'' agree on the second
argument. Similarly the clauses $\eqalLF$ seems to share subexpressions
with $\eqilLF$ and $\eqirLF$. To insert these four clauses into a
substitution tree, we need to first translate them into linear
higher-order patterns. Although all the terms in these clauses fall
into the pattern fragment, not all of them are linear patterns.
%since all the
%existential variables are applied to distinct variables, 
%not all of them are applied to {\em{all}} distinct bound variables. 
For example, in clause $\eqilLF$, 
%\[
%\eqLF (\impLF (\existsLF \lambda x. A\; x)\; B)\quad (\forallLF
%\lambda x. (\impLF (A\; x)\; B)
%\]
%
the existential variable $B$ does not depend on the bound variable
$x$, in $(\forallLF \lambda x. (\impLF (A\; x)\; B)$. Hence, $B$ is
not a linear higher-order pattern, since it is not applied to all
bound variables in whose scope it occurs. In addition, the existential
variable $A$ occurs twice. Before inserting the clause heads into a
substitution tree, we linearize them by eliminating any duplicate
occurrences of existential variables, and replacing any existential
variable which is not fully applied with one which is. The linearized
program  is given next:


\begin{small}
\[
\begin{array}{ll}
% \eqLF: \quad \propLF \rightarrow \propLF \rightarrow \typeLF.\\[1em]
%
\eqilLF: \eqLF (\impLF (\existsLF \lambda x. A\; x)\; B)\;
                 (\forallLF \lambda x. (\impLF (A'\; x)\; (B'\;x))). \\
\hspace{1cm}\forall x. (A'\; x) \unif (A \; x) {\textsf{ and } } B'\;x   \unif B\\[0.5em]
\eqirLF: \eqLF (\impLF A\; (\forallLF \lambda x. B\;x))\quad
                 (\forallLF \lambda x. (\impLF (A'\;x) \; (B'\; x))).\\
\hspace{1cm} \forall x. (A'\; x) \unif A  {\textsf{ and }} B'\;x   \unif (B\;x)\\[0.5em]
%\eqarLF: \eqLF (\andLF (\forallLF \lambda x. A \;x) \; B)\quad
%                 (\forallLF \lambda x. \andLF (A'\; x) \; (B'\;x)).\\
%\hspace{1cm} \forall x. (A'\; x) \unif (A \; x) {\textsf{ and }} B'\;x   \unif B\\[0.5em]
\eqalLF: \eqLF (\andLF A \; (\forallLF \lambda x. B\; x)) \quad
                 (\forallLF \lambda x. \andLF (A'\;x) \; (B' x)).\\
\hspace{1cm} \forall x. (A'\; x) \unif A  {\textsf{ and }} B'\;x   \unif (B\;x)\\[0.5em]
\end{array}
\]
\end{small}

Now even more sharing becomes apparent. For example, the clauses
$\eqilLF$ and $\eqirLF$ agree upon the last argument. Similarly, the
clauses $\eqalLF$ and $\eqarLF$. We now compute the most specific
generalization between these clauses, and can build up a substitution
tree. The algorithm for computing the most specific generalization is
given in \cite{Pientka03phd,Pientka:ICLP03}.


\begin{figure*}[htbp]
  \begin{center}
    \begin{small}
% nodesep=1pt,
\pstree[nodesep=0.5pt,levelsep=8ex]{%
\TR{$\eqLF\quad i_2 \quad (\forallLF \lambda x. i_1\;x)$} }{%
  \pstree{\TR{\begin{tabular}{r}
              $\lambda x.(\andLF (A'\;x)\; (B'\;x))/i_1$\\
%              $(\andLF (i_3\;x)\; (i_4 \;x))/i_2$\\
              $(\andLF A\;\;(\forallLF \lambda x. (B\;x)))/i_2$\\
            \end{tabular}
          }
        }{%
%          \pstree{\TR{\begin{tabular}{r}
%                $\lambda y.(\forallLF \lambda x. A\;x)/i_3$,\\
%                ${\tt \lambda y.B/i_4}$
%              \end{tabular}
%            }}{%
%            \pstree{\TR{\begin{tabular}{l}
%                  $\forall x. A'\;x\unif A\;x$\\
%                  $B'\;x\unif B$
%                \end{tabular}
%              }}{ $\eqarLF$}
%          }
%
%          \pstree{\TR{\begin{tabular}{r}
%                ${\tt \lambda y.A/i_3},$\\
%                $\lambda y.(\forallLF \lambda x. (B\;x))/i_4$
%              \end{tabular}
%            }
%          }{%
            \pstree{\TR{\begin{tabular}{l}
                $\forall x. A'\;x\unif A$\\
                $\forall x. B'\;x \unif B\;x$
              \end{tabular}
            }
            }{$\eqarLF$}
%          }
        }
%%%%% second child
          \pstree{\TR{\begin{tabular}{r}
                     $\lambda x.(\impLF (A'\;x)\; (B'\;x))/i_1$\\
                     $(\impLF (i_3\;x)\; (i_4\;x))/i_2$\\
                   \end{tabular}
                 }}{
                 \pstree{\TR{\begin{tabular}{r}
                       $\lambda y.\forallLF \lambda x. A\;x/i_3$,\\
                       ${\tt \lambda y.B/i_4}$
                       \end{tabular}
                     } }{%
                     \pstree{\TR{\begin{tabular}{l}
                           $\forall x. A'\;x\unif A\;x$\\
                           $\forall x. B'\;x\unif B$
                         \end{tabular}}
                     }{$\eqirLF$}
                   }
                  \pstree{\TR{\begin{tabular}{r}
                        ${\tt \lambda y.A/i_3},$\\
                        $\lambda y.(\forallLF \lambda x.B\;x)/i_4$
                      \end{tabular}}}{
                    \pstree{\TR{\begin{tabular}{l}
                          $\forall x. A'\;x \unif A$\\
                          $\forall x. B'\;x \unif B\;x$
                        \end{tabular}}
                    }{$\eqilLF$}
                  }
                }
              }     
    \end{small}
  \end{center}
  \caption{Substitution tree}
  \label{fig:substree}
\end{figure*}

% \end{small}

By composing the substitutions along a path,
we will obtain a clause head. By composing the substitutions in the
left-most branch, we obtain the clause head $\eqalLF$.  
In contrast to other indexing techniques such as discrimination tries,
substitution trees  allows the sharing of common sub-expressions
instead of common term prefixes. As we can see in this example, this
is especially useful in this example, since the most sharing is done
in the second argument. 

We have chosen to index only the static set of program clauses. In
theory, it is possible to use substitution tree indexing for dynamic
clauses generated during proof search.  However, it is not clear how
useful this will be, since the process of creating the tree itself is
time-consuming. It is also noted by Necula and Rahul
\cite{Necula+01:oracle} that indexing dynamic assumptions imposes a
performance penalty. It is useful to pre-process the program, but the
payoff with dynamic clauses is unclear. For dynamic clauses, we use
only simple indexing on the type family of type.

%\begin{note}
%  \begin{itemize}
%  \item Should we talk about lowered terms and modal variables
%  \item Should be give an algorithm for computing the most specific
%    generalization? (it is given in ICLP'03)
%  \end{itemize}
%\end{note}


\subsection{Caching results}
\label{sec:tabling}
Since large proofs often have identical subproofs,  there is a
lot of potential for sharing subproofs. The problem is particularly
acute in machine-generated proofs for certifying machine-code which
tend to have repeated proofs of simple facts. This problem has been
already pointed out by Necula and Lee in \cite{NeculaLee+97:resource}

\begin{quote}
% \begin{tabular}[h]{l}
``...it is very common for the proofs to have  
repeated sub-proofs that should be hoisted out and 
proved only once ...'' \cite{NeculaLee+97:resource}
 \end{quote}


In the context of generating and checking small proof witnesses, this
leads to two problems.  First, the proof witnesses become larger in
size than necessary. This means that what has to be transmitted to the
verifier is large in size. Secondly, the performance of witness
checker may degrade, since it spends its time uselessly proving the
same fact over and over again. Ideally we would like to cache
intermediate results and re-use them later. In this section, we
describe an extension to generating and checking proof witnesses with
caching. The idea can be briefly summarized as follows: when
generating compressed witnesses for a goal $G$ from explicit proof
terms, we store intermediate goals together with their solutions in a
table, and re-use the result later on. For checking proof witnesses,
we consult the oracle to see whether the tabled answer is to be
used. Further, intermediate goals are stored similarly to the
compression case.

Since caching everything is too overwhelming in most cases in
practice, hence we support selective caching. The user declares
certain  predicates to be cached, and for those we will factor out
common sub-proofs. We will modify the {\sf{Select}} step in our
previous search procedure to allow for caching by defining an
auxiliary procedure {\sf CallCheck}. {\sf{CallCheck($\Gamma \vd G, \cal{T}$)}} checks whether a variant of
the current subgoal exists or if the current subgoal is an instance of
a previous entry. If there exists a table entry $\Gamma' \vd G'$
s.t. $\Gamma \vd G$ is a variant (or instance) of the already existing
entry $\Gamma' \vd G'$, then a pointer to the corresponding answer
list is returned. If no such entry exists, $\Gamma \vd G$ is added to
the table $\cal{T}$ and a pointer to an empty answer list is
returned. If no entry exists, then we will continue to focus on a
clause $c_i$ to solve the goal $\Gamma \vd G$. When we are done, we
will add the answer substitution for the existential variables in
$\Gamma \vd G$ together with its proof term $c_i \cdot S$ to the its
answer list $ptr$. 

If a table entry for $\Gamma \vd G$ already exists, there are two
possible situations: 1) If the answer list contains an answer
substitution $\theta_k$  which leads to a proof $c_i \cdot S$,
then we will just re-use the answer substitution $\theta_k$. 2) If the
answer list does not contain an answer which would lead to a proof
$c_i \cdot S$, then we need to use a program clause $c_j$ to focus on
and solve the goal $\Gamma \vd G$. The witness generation using
caching can be summarized as follows:

%\begin{figure}[h]
%\fighead
\begin{center}
%\parbox[h]{12cm}{
%Witness generation using caching}
\begin{small}
% \noindent \mbox{{\bf{Solve Goal $\Gamma \vd M: G$:}}\hfill}% \\[-2.5em]
\begin{description}
\item[Select-Cache] $\Gamma \vd c_i \cdot S: G \Rightarrow
  \underset{1 \ldots (j-1)}{\underbrace{0\ldots 0}}1W $ \\
    \mbox{Given an atomic goal $G$, clauses $\Gamma$, and a table $\cal{T}$:}\hfill
\\
    {\em{if}} CallCheck($\Gamma \vd G$, $\cal{T}$) \\
         {\em{then}} return a pointer ptr to the answer list\\
        \hspace{0.5cm}{\em{if}} answer($c_i \cdot S$, ptr) \\
        \hspace{0.75cm}{\em{then}} return $\theta_j$ where $(c_i \cdot S,
        \theta_j)$ is j-th answer in the answer list and \\
        \hspace{1.5cm}continue to solve any open subgoals under
         $\theta_j$ \\
         \hspace{0.75cm}{\em otherwise}\\
         \hspace{1cm}Let  $k''$ be the length of the answer list ptr
         and $c_i : A_i$  be the\\
         \hspace{1cm} i-th clause which leads to a
         proof $c_i\cdot S$ of $G$ from 
         $\Gamma$; then $j = k'' + i$\\
         \hspace{1cm}Focus on clauses $c_i : A_i$ from $\Gamma$ to compress 
         a proof $c\cdot S$ for $G$ to $W$ and \\
         \hspace{1cm}add the answer substitution together with
         $c_i\cdot S$ to the answer list ptr\\ 
         % its actually not the proof term we store....would also be
% too much... -bp
      {\em otherwise} \\
\hspace{0.5cm}Let  $c_j : A_j$ be the j-th clause which leads to a 
         proof $c_i\cdot S$ of $G$ from $\Gamma$;\\
\hspace{0.5cm}Focus on clauses $c_j : A_j$ from $\Gamma$ to compress 
a proof $c\cdot S$ for $G$ to $W$.\\
     \hspace{0.5cm}Add the answer substitution together with
         $c_i\cdot S$ to the answer list ptr\\[1em]

%\item[CheckGoal]($\Gamma \vd c_i \cdot S: G$)\\
%    Check if this goal $\Gamma \vd G$ is already in the table $\cal{T}$\\
%    \hspace{0.5cm}If Yes, return the list of answers present\\
%    \hspace{0.5cm}If No, add goal to the cache with \\
%    \hspace{0.7cm} initially null answer list,
%return a pointer to this list\\
\end{description}
%   \caption{Solve goal $G$ from clauses in $\Gamma$}
\end{small}    
%
\end{center}
%\figfoot
%\caption{\label{fig:pwcache} Proof Compression with Caching}
%\end{figure}


The generation and checking of witnesses will follow similar
algorithms, so both have identical caches and consider the same number
of choices. 
%Hence, we just need a
%convention on which order to consider tabled answers in, and we choose
%the choices appearing in the table to be tried later than the other choices.
%In the following, we will highlight some of the important
%invariants underlying our implementation. 
%%
%%Up to now, this has been difficult since we need to efficiently
%%store and retrieve intermediate goals. We adapt and build upon recent
%%work \cite{Pientka03phd} on memoizing intermediate goals during
%%execution as part of the tabled higher-order logic programming and
%%adapt it for witness generation and witness checking. 
%%
%%A naive implementation can result in repeatedly rescanning terms and
%%thereby degrading performance considerably. 
%Typically intermediate goal
%$G$ may contain existential  variables which are realized via
%references and destructive updates in an implementation. This achieves
%that instantiations of existential variables are immediate.On the
%other hand the state of the existential variables may change during
%proof search. Hence when tabling a given subgoal, we abstract over all the
%existential variables and store an abstract version of the
%subgoal to avoid the pollution of the table entries. A similar problem
%arises when adding answer substitutions, since answers may not
%necessarily be ground.
%%Hence before adding a subgoal with existential variables, we will
%%abstract over them and standardize the goal together with its dynamic
%%assumptions. 
%%
%In general, the invariant about table entries and answer substitutions
%are:
%%can be summarized as follows: 
%\[
%\begin{array}{ll}
%\mbox{Table entry}\quad\quad & \quad\mbox{Answer substitution}\\
%\Delta ; \Gamma \vd G & \quad \Delta' \vd \theta : \Delta
%\end{array}
%\]

%$\Delta$ refers to a context describing existential variables,
%$\Gamma$ describes the context for the bound variables  and dynamic
%assumptions and $G$ describes the goal we are trying to prove. 
%%In addition, we translate every subgoal into a linear higher-order pattern
%%together with some residual constraints $R$
%The design supports naturally substitution factoring based on explicit
%substitutions\cite{RamakrishnanJLP99}. With substitution factoring the
%access cost is proportional to the size of the answer substitution
%rather than the size of the answer itself. It guarantees that we only
%store the answer substitutions, and create a mechanism of returning
%answers to active subgoals that takes time linear in the size of the
%answer substitution $\theta$ rather than the size of the solved query
%$[\theta]G$. In other words, substitution factoring ensures that answer
%tables contain no information that also exists in their associated
%call table. Operationally, this means that the constant symbols in the
%subgoal need not be examined again when we insert answer
%substitutions. For this setup to work cleanly in the higher-order
%setting, it is crucial that we distinguish between existential
%variables in $\Delta$ and bound variables and assumptions 
%in $\Gamma$. 

%To allow easy comparison of goals $G$ with dynamic assumptions
%$\Gamma$ modulo renaming of existential variables and bound variables, we
%represent terms internally using explicit substitutions
%\cite{Abadi:POPL90} and de Bruijn indices. The basic underlying idea
%is to use a nameless representation of variables based on de Bruijn
%indices. The main advantage is that equality up to renaming
%of bound variables is reduced to syntactic equality checks, if all
%objects are in $\beta\eta$-normalform.

%Storing large sets of intermediate goals together with their proofs
%can only be practical if the table operations are efficient and
%exploit sharing common structure and common operations. We use
%higher-order substitution tree indexing from the previous section to
%store intermediate goals $G$ together with their dynamic assumptions
%$\Gamma$ and the corresponding proof $c_i\cdot S$\footnote{We only
%  store a skeleton of   the proof term.}. This requires however that
%goals are in linear form, i.e. all existential variables are fully
%applied. Therefore, we linearize goals $\Gamma \vd G$ during
%abstraction and factor out any non-linear sub-expressions.
%%
%%There are differences from the usual use of the table in tabled logic
%%programming. In tabled logic programming, if we encounter a goal which
%%is in the table, but whose answer we have not found yet, we suspend
%%execution. There is a notion of staging, and the goal will be
%%reactivated in the next stage, when we may have found the answer. This
%%is important for completeness of tabled logic programming. In proof
%%compression or checking, we do not care about retrieving all possible
%%proofs, just the particular proof we are interested in. Thus execution
%%is never suspended for later stages, since we know that the current
%%goal is always solvable. We will continue solving in the above situation.
%%
%%
% This motivates the final table design: 

%\[
%\begin{array}{lll}
%\mbox{Table entry}\quad \quad &\quad \mbox{Residual Equ.}\quad &\quad \mbox{Answer substitution}\\
%\Delta ; \Gamma \vd G & \quad \Delta ; \Gamma \vd R  & \quad \Delta' \vd \theta : \Delta
%\end{array}
%\]

%where $G$ is a linear higher-order pattern, $\Gamma$ denotes the bound
%variables and dynamic assumptions and $\Delta$ describes the
%existential variables occurring in $G$ and $\Gamma$. This
%linearization step can be done together with abstraction and 
%standardization over the existential variables in goal, hence only one
%pass through the term is required. 


%\subsection{Optimization: Strengthening and subsumption}

%Apply strengthening to detect more identical subproofs (strengthening
%based on subordination and on not adding dynamic assumption twice)

%\begin{note}
%  Say more about strengthening; is it critical in the proofs about the
%  sequent calculus?  

%  I don't know how important strengthening is in the context of this work.
%  Can we come up with an example where it demonstrably helps? - Susmit
%\end{note}

\section{Experimental Results}

In this section, we give an experimental evaluation of generating and checking 
compact proof witnesses. In particular, the results discuss the
trade off between witness size and the time it takes to construct or check
witness. Our comparison will be threefold: First, we will consider
finding a proof via proof search thereby constructing a proof without
any guidance, i.e. the witness size is zero. The downside of this approach is
that we cannot always reconstruct the full proof automatically and we
need to trust the a potentially large and complicated implementation
of the higher-order logic programming engine.  Second, we will
consider proof checking of explicit proof terms. If we send the
explicit proof term then then checking the proof will always succeed
and the proof checker is only a few hundred lines of code. However,
the size of the explicit proof term is substantial. Finally, we will
consider our approach of small proof witnesses, which only keeps track
of minimal information to allow the proof to be re-run by a
deterministic logic programming interpreter. As a consequence, 
our higher-order logic programming engine is more light-weight, since
we do not need to backtrack or trail existential variables. It
provides a small proof witness and allows for fast witness checking,
although the code base to be trusted is still quite larger.
Finally, we will discuss the trade-offs of caching subproofs and compare
different encoding schemes for describing the non-deterministic choices. 
Also, we give a study of how the encoding scheme changes the sizes of the 
oracles.

Our experiments are run on a Pentium 4 machine with 1 GB of memory size.
The machine runs Twelf compiled by SML of New Jersey version 110.0.7, and
runs it on the Redhat Linux 7.1 operating system, with no other programs
running on the background. We present a representative selection of results
from an extensive suite of experiments we have run.
   
%Further, since 
%the primary focus is on generating small proof witnesses, we will
%study how the size of the proof witnesses change by varying
%parameters. We hope this will serve as a guide to those wishing to 
%generate and use oracles on the various tradeoffs to be made.

\subsection{Sequent Calculus}
Our first example suite is an implementation of a sequent calculus 
for intuitionistic propositional logic where invertible rules are
chained together thereby eliminating some non-determinism in the
overall proof search. 
%This has the standard connectives
%for conjunction, disjunction, implication and atomic propositions 
%truth and falsehood. The calculus is one designed to find proofs using an 
%inversion based strategy.

\begin{table*}[htbp]
\begin{center}
\begin{small}
\begin{tabular}{|l|c|c|c|c|c|c|c|c|}
\hline
Example & PST & WV & (PST/WV) & PS & PST & WS & (PS/WS)\\
\hline
$(A\supset B)\wedge (A\supset C)\Rightarrow A\supset B\wedge C$
&       0.47 
&     $<$  0.01 
& $\infty$
&       361 
&       43 
&       5 
&       72.2\\
%% $(A\supset B)\vee (A\supset C)\Rightarrow A\supset B\vee C$
%% &       0.380 
%% &       0.010 
%% &       0.000 \\
$A\vee C\wedge (B\supset C)\Rightarrow (A\supset B)\supset C$
&       1.70 
&       0.01
&       170 
&       570 
&       50 
&       6 
&       95\\
$(A\supset C)\wedge (B\supset C)\Rightarrow A\vee B\supset C$
&       2.18 
&      $<$ 0.01
&      $\infty$ 
&       561 
&       56 
&       6 
&       93.5\\
%% $A\wedge B\vee C\Rightarrow (A\wedge B)\vee (A\wedge C)$
%% &       0.190 
%% &       0.020 
%% &       0.000 \\
%% $(A\wedge B)\vee (A\wedge C)\Rightarrow A\wedge B\vee C$
%% &       0.760 
%% &       0.010 
%% &       0.000 \\
%% $A\vee (B\wedge C)\Rightarrow A\vee B\wedge A\vee C$
%% &       0.210 
%% &       0.020 
%% &       0.000 \\
%% $A\vee B\wedge A\vee C\Rightarrow A\vee (B\wedge C)$
%% &       0.790 
%% &       0.020 
%% &       0.000 \\
%% $\Rightarrow (A\supset B)\wedge (A\supset C)\supset A\supset B\supset C$
%% &       0.390 
%% &       0.020 
%% &       0.000 \\
$\Rightarrow (A\supset C)\wedge (B\supset C)\supset A\vee B\supset C$
&       2.43 
&       0.01
&       243 
&       792 
&       57 
&       6 
&       132\\
%% $\Rightarrow (A\supset B)\vee (A\supset C)\supset A\supset B\vee C$
%% &       0.480 
%% &       0.010 
%% &       0.000 \\
%% $\Rightarrow A\wedge (B\supset \perp)\supset (A\supset B)\supset \perp$ 
%% &       0.370 
%% &       0.020 
%% &       0.000 \\
\hline
\end{tabular}
\begin{tabular}{ll@{=}ll@{=}l}
Key & PST & Proof Search time 
&WV & Witness Verification time \\ 
&PS & Proof Size in bytes\\
&PST & Proof Size in Number of tokens 
&WS & Witness Size in bytes\\
\end{tabular} 
\end{small}
\end{center}
\caption{\label{tab:seqtimes}
Sequent Calculus: Times with Caching of User-Selected Predicates}
\end{table*}

We included the time to find a proof to contrast it against the proof
checking times. We use the tabled higher-order logic programming
engine \cite{Pientka05,Pientka03phd} to find proofs for the
propositional logic. The proof compression and the verification procedures
provide significant time speedups, since in these procedures, we already
know the proof. This is exhibited in Table~\ref{tab:seqtimes}.

%% \begin{table*}[htbp]
%% \begin{center}
%% \begin{small}
%% \begin{tabular}{|l|l|l|l|}
%% \hline
%% Example & Proof Size & Proof Size & Witness Size \\
%% & (bytes) & (number of tokens) & (bytes)\\
%% \hline
%% $(A\supset B)\wedge (A\supset C)\Rightarrow A\supset B\wedge C$
%% &       361 
%% &       43 
%% &       5 \\
%% %% $(A\supset B)\vee (A\supset C)\Rightarrow A\supset B\vee C$
%% %% &       292 
%% %% &       45 
%% %% &       5 \\
%% $A\vee C\wedge (B\supset C)\Rightarrow (A\supset B)\supset C$
%% &       570 
%% &       50 
%% &       6 \\
%% $(A\supset C)\wedge (B\supset C)\Rightarrow A\vee B\supset C$
%% &       561 
%% &       56 
%% &       6 \\
%% %% $A\wedge B\vee C\Rightarrow (A\wedge B)\vee (A\wedge C)$
%% %% &       280 
%% %% &       37 
%% %% &       6 \\
%% %% $(A\wedge B)\vee (A\wedge C)\Rightarrow A\wedge B\vee C$
%% %% &       437 
%% %% &       66 
%% %% &       7 \\
%% %% $A\vee (B\wedge C)\Rightarrow A\vee B\wedge A\vee C$
%% %% &       331 
%% %% &       50 
%% %% &       6 \\
%% %% $A\vee B\wedge A\vee C\Rightarrow A\vee (B\wedge C)$
%% %% &       428 
%% %% &       47 
%% %% &       6 \\
%% %% $\Rightarrow (A\supset B)\wedge (A\supset C)\supset A\supset B\supset C$
%% %% &       454 
%% %% &       44 
%% %% &       5 \\
%% $\Rightarrow (A\supset C)\wedge (B\supset C)\supset A\vee B\supset C$
%% &       792 
%% &       57 
%% &       6 \\
%% %% $\Rightarrow (A\supset B)\vee (A\supset C)\supset A\supset B\vee C$
%% %% &       440 
%% %% &       46 
%% %% &       5 \\
%% %% $\Rightarrow A\wedge (B\supset \perp)\supset (A\supset B)\supset \perp$ 
%% %% &       513 
%% %% &       38 
%% %% &       5 \\
%% \hline
%% \end{tabular}
%% \end{small}
%% \end{center}
%% \caption{\label{tab:seqsizes}
%% Sequent Calculus: Sizes with Caching of User-Selected Predicates}
%% \end{table*}

Next, we turn our attention to questions of proof size. 
Table~\ref{tab:seqtimes} compares the size of our proof witnesses to the
size of the original proof. The original proof is measured both by number 
of bytes as well as the number of tokens. These figures are assuming 
caching is done of predicates the user has selected, and the encoding of
the oracle is in terms of the unary encoding scheme described earlier.

\subsection{Refinement Types as an advanced type system}
Our next example is an advanced type system for a high-level
call-by-value functional language. The language has functions, a
fixpoint construct, booleans and bitstrings. The type system for the
language has refinement types, as described in Davies and
Pfenning~\cite{davies+:intersection}. In particular, the type of
bitstrings is refined by zero and strictly positive number
representations.

Table~\ref{tab:reftimes} demonstrates that proof checking yields a
speedup between 2 and 6 times. This figure is achieved if we are caching
subgoals to get maximum compressions. As we see later, even more gains
can be achieved by turning off caching. 
%% Turning the cache off however
%% would overstate the gains, since proof search has to use tabling to
%% find the proof\footnote{-bp : this should be executable without
%%   tabling; to be more precise, the benefit of tabling is really that
%%   we get consistent performance, no matter if we prove or disprove a
%%  query in these examples.}.
This gain comes about since we do not have to explore unproductive
branches of the proof tree.

We also compare proof size to the size of the compact witness we
produce in Table~\ref{tab:reftimes}. We notice that the compact oracle
is about 1 \% of the size of the proof term.

\begin{table*}[htbp]
\begin{center}
\begin{small}
\begin{tabular}{|l|c|c|c|c|c|c|c|}
\hline
Example & PST 
& WV & ( PST / WV ) & PS & PSN & WS & (PS / WS)\\
\hline
mult-pos-nat & 5.81 & 1.10 & 5.28
& 15654 & 1159 & 169 & 92.62\\
%%mult-3-types & 8.42 & 1.60 & 1.49 & 5.65 \\
mult & 0.39 & 0.13 & 3.0 
& 6074 & 509 & 47 & 129.23\\
%%square & 0.42 & 0.16 & 0.13 & 2.62 \\
square-pos-nat & 12.55 & 1.85 & 6.78 
& 25303 & 1587 & 242 & 104.55\\
%%square-pos-pos & 11.880 & 1.920 & 2.130 & 6.18 \\
\hline
\end{tabular}
\begin{tabular}{ll@{=}ll@{=}l}
Key & PST & Proof Search time
&WV & Witness Verification time\\ 
&PS & Proof Size in bytes\\
&PST & Proof Size in Number of tokens 
&WS & Witness Size in bytes\\
\end{tabular} 
\end{small}
\end{center}
\caption{\label{tab:reftimes} Refinement Type System : 
Proof Compression Times with Caching}
\end{table*}

%% \begin{table*}[htbp]
%% \begin{center}
%% \begin{small}
%% \begin{tabular}{|l|l|l|l|l|l|}
%% \hline
%% Example & Proof Term & Proof Term & Witness & Proof Size\\
%% & Size (Bytes) (1) & Size (Tokens) & Size (Bytes) (2) & percentage (2 / 1)\\
%% \hline
%% mult-pos-nat & 15654 & 1159 & 169 & 1.07 \%\\
%% %%mult-3-types & 18537 & 1383 & 211 & 1.13 \%\\
%% mult & 6074 & 509 & 47 & 0.77 \%\\
%% %%square & 7060 & 546 & 50 & 0.70 \%\\
%% square-pos-nat & 25303 & 1587 & 242 & 0.95 \%\\
%% %%square-pos-pos & 24957 & 1560 & 237 & 0.94 \%\\
%% \hline
%% \end{tabular}
%% \end{small}
%% \end{center}
%% \caption{\label{tab:refsizes} Refinement Type System : 
%% Size of Witness with Caching}
%% \end{table*}

Next we investigate the practicality of caching subgoals. Caching is 
a fairly expensive operation, in terms of both time to store and lookup, 
as well as the extra memory required to maintain the table. In 
Table~\ref{tab:refcache} we investigate this trade off. We find that
using caching gives us a speed hit of between 3 and 16 times. The 
gains from this is that the size of the oracle is smaller for the
cached version, for every experiment in this set. The gain is small
but significant.

\begin{table*}[htbp]
\begin{center}
\begin{small}
\begin{tabular}{|l|c|c|c|c|c|c|c|}
\hline
Example & \multicolumn{3}{c}{Compression Time} & 
\multicolumn{3}{c}{Witness Size} & Table\\
& Cached & Uncached & Slowdown & Cached & Uncached & Saving & Size\\
& (s) (1) & (s) (2) & (1 / 2) & (bytes) (3) & (bytes) (4) & (4 - 3) / 4& \\
\hline
mult-pos-nat & 1.18 & 0.11 & 10.72 & 169 & 171 & 1.16 \% & 579\\
%%mult-3-types & 1.60 & 0.41 & 3.90 & 211 & 216 & 2.31 \% & 691\\
mult & 0.140 & 0.05 & 2.8 & 47 & 67 & 29.85 \% & 164\\
%%square & 0.16 & 0.04 & 4.0  & 50 & 71 & 29.57 \% & 179\\
square-pos-nat & 2.31 & 0.16 & 14.43 & 242 & 247 & 2.02 \% & 794\\
%%square-pos-pos & 1.92 & 0.16 & 12.0 & 237 & 243 & 2.46 \% & 775\\
\hline
\end{tabular}
\end{small}
\end{center}
\caption{\label{tab:refcache} 
Refinement Type System : 
Caching during proof compression}
\end{table*}

\subsection{Foundational Proof Carrying Code}
Our last example suite is an implementation from the Foundational 
Proof Carrying Code project at Princeton~\cite{Appel01lics}. This 
is a large program that type checks SPARC object code with the help 
of annotations produced by a compiler. The type system used is a
low-level type system known as LTAL~\cite{chen+:fpcc-ltal}.

\begin{table*}[htbp]
\begin{center}
\begin{small}
\begin{tabular}{|l|c|c|c|c|c|c|c|}
\hline
Example & PST & WV & (PST / WV) & PS & PSN & WS & (PS / WS)\\
& (s) & (s) & & (bytes) & (tokens) & (bytes) &\\
\hline
clos & 12.26 & 0.47 & 26.08& 201910 & 16502 & 638 & 316.47\\
mid & 10.29 & 0.45 & 22.86& 398589 & 34250 & 528 & 754.90\\
inc & 11.55 & 0.47 & 24.57& 410600 & 35724 & 579 & 709.15\\
lint & 12.84 & 0.70 & 18.34& 441965 & 38416 & 703 & 628.68\\
\hline
\end{tabular}
\begin{tabular}{ll@{=}ll@{=}l}
Key & PST & Proof Search time
&WV & Witness Verification time\\ 
&PS & Proof Size in bytes\\
&PST & Proof Size in Number of tokens 
&WS & Witness Size in bytes\\
\end{tabular} 
\end{small}
\end{center}
\caption{\label{tab:fpcctimes}
FPCC: Times without Caching}
\end{table*}

In Table~\ref{tab:fpcctimes} we show the gains to be achieved in terms of
time performance. In the proof carrying code scenario, asking the consumer
to verify our compact proof witness as opposed to the proof term gives a 
speedup of about 20 times. Also important is the size of the proof that
must be sent to the consumer. Our proof witnesses are between 300 and 700
times smaller than the corresponding proof terms, as we show in
Table~\ref{tab:fpcctimes}.

%% \begin{table*}[htbp]
%% \begin{center}
%% \begin{small}
%% \begin{tabular}{|l|l|l|l|l|}
%% \hline
%% Example & Proof Size & Proof Size & Witness Size & Gain \\
%% & (bytes) (1) & (number of tokens) & (bytes)(2) & (1 / 2)\\
%% \hline
%% clos & 201910 & 16502 & 638 & 316.47\\
%% mid & 398589 & 34250 & 528 & 754.90\\
%% inc & 410600 & 35724 & 579 & 709.15\\
%% lint & 441965 & 38416 & 703 & 628.68\\
%% \hline
%% \end{tabular}
%% \end{small}
%% \end{center}
%% \caption{\label{tab:fpccsizes}
%% FPCC: Sizes without Caching}
%% \end{table*}

Finally, we study the issue of unary versus binary encodings of the choices.
A representative study with examples from multiple example suites is given
in Table~\ref{tab:unarybinary}. We notice that binary encodings always 
increase the size of the oracle, by between 7\% and 115\%. As we discussed
before, logic programmers usually write their programs so that the first
few clauses are the ones that are used more commonly, in which case unary 
encodings are better.

\begin{table*}[htbp]
\begin{center}
\begin{small}
\begin{tabular}{|l|c|c|c|}
\hline
Example & WSU & WSB & (WSB - WSU/ WSU) \\
\hline
clos & 638 & 715 & 12.0 \%\\
mid & 528 & 652 & 23.5 \%\\
%%inc & 579 & 678 & 17.1 \%\\
lint & 703 & 754 & 7.3 \%\\
mult-pos-nat & 171 & 338 & 97.7 \%\\
mult & 67 & 144 & 114.9 \%\\
%%square-pos-nat & 247 & 475 & 92.3 \%\\
\hline
\end{tabular}
\begin{tabular}{ll@{=}l}
Key & WSU & Witness Size in bytes for Unary Encoding\\
&WSB & Witness Size in bytes for Binary Encoding\\
\end{tabular} 
\end{small}
\end{center}
\caption{\label{tab:unarybinary}
Unary versus Binary Encodings: no Caching}
\end{table*}

\section{Related Work}
The idea of compact proof witnesses which encode the
non-deterministic choice in a logic programming interpreter was first
proposed by \cite{Necula+01:oracle} for a fragment of the logical
framework LF, called LF$_i$, which excludes the use of higher-order
terms and dependent types in practice. 
Their main goal was to design a practical method for current
proof-carrying code applications which would reduce the size of proofs
sent to a consumer. To achieve this goal, their approach
has been very pragmatic. For example they restrict themselves in
practice to first-order terms and and significantly limit the use
of dependent types. To achieve an efficient
implementation, they propose the use of automata-driven indexing,
where any higher-order features are ignored. Their indexing algorithm
will generate a set of potential candidates of which unsound
candidates need to be weeded out by calling full 
higher-order unification based on Huet's algorithm. This is clearly
wasteful and expensive in the general higher-order case, since we will
traverse higher-order terms at least twice. Moreover, since they use
Huet's unification algorithm, which is non-deterministic itself, 
their proof witnesses need to record the choices made within
higher-order unification. To avoid these problems in practice, their
realization and their experimental evaluation does not consider terms
defined via $\lambda$-abstraction.  %To handle some simple cases
%like the {\tt alli} rule in our example, they introduce a
%hack\ednote{-bp: do you know what hack they use?} to deal
%with this special case. 
Our work extends and continues where Necula and Rahul left of by
saying ``more experimental results are needed especially in the
higher-order setting''. By using linear higher-order patterns and
higher-order substitution tree indexing, the non-determinism based on
Huet's unification algorithm is removed. Moreover, our implementation scales
to dependent types thereby removing any restrictions imposed by
Necula and Rahul. In addition, we designed an extension where
sub-proofs can be cached and re-used resulting in potentially even
smaller proof witnesses.

The idea of using oracles was also explored in
\cite{Appel:PPDP03}. Their primary concern was to 
achieve a minimal trusted proof checker. Their checker follows the
path explored by Necula and Rahul and ignores higher-order terms. The
main difference between the two approaches is that the proof rules are
proven correct independently thereby minimizing the trusted computing
base. Trust is not our concern here, rather we aim at extending the
safety infrastructure already provided by Twelf with capabilities of
generating and checking small proof witnesses. This step, we believe,
will provide the developers of safety policies in Twelf with new
insights about the relationship or safety rules and size of proofs.

As Necula and Rahul's work, their system is not able to support
higher-order abstract syntax, which means that that any variable
binding constructs must be explicitly encoded. Wu {\em et
  al}\cite{Appel:PPDP03} encode the explicit substitution calculus
\cite{Abadi:POPL90} together with the necessary proofs about
substitutions for their foundational implementation of LTAL. Although
the overhead in this setting is still manageable, it is not general
enough to handle richer safety polices. 

% Our work fills this gap by describing a general purpose tool to
% compress proofs to compact witnesses and check witnesses within the logical
%framework LF. We believe this is a valuable addition to the general
%safety infrastructure already provided by Twelf, which allows users to
%gain insights into the relationship between safety policies and small
%proofs. 

%\begin{note}
%  Susmit: say more about the difference and why ours is better.

%-- Not sure what you want me to say. Is it not using oracles during
%   higher-order unification? - Susmit

%-- bp: I thought you are also generating the bits in a different
%fashion. I thought Necula encoded each choice differently. For
%example, one can encode pick the 3rd clause out of 4 as 0010
%or as just 3=11. I remember some discussion with Karl, where you
%suggested that a scheme which encodes 4 as 0010 is better although it
%may produce larger bit-strings sometimes? -- Sofar I think it is still
%a bit unclear *how* we actually encode the choices -- in other words
%how is 3/4 encoded.

%\end{note}


%\begin{note}
% 1. Should we talk about justifiers? - Susmit Yes. Especially if we
% want to send it to ICLP. -bp see Abhik Roychoudhury and C. R. Ramakrishnan and I. V. Ramakrishnan,
%    "Justifying proofs using memo tables", there is a more recent
%    journal version.\\
% 2. Do we need to describe LF$_i$? The references to a 2-level framework 
%    may be incomprehensible to most people. On the other hand, it is not
%    directly relevant - Susmit
%\end{note}

%It is also worth pointing out that in the higher-order setting the
%choice of unification algorithm is crucial since not all unification
%algorithms are deterministic. Higher-order unification based on Huet's
%algorithm \cite{Huet75} for example is highly non-deterministic. On the
%other hand higher-order pattern unification \cite{Miller91jlc} is
%decidable and deterministic for higher-order patterns. Although it is too restrictive to
%concentrate solely on higher-order patterns statically, most of the
%non-patterns encountered (for example $(A\;T)$), will become patterns
%during execution. Twelf uses higher-order pattern unification together
%with constraints \cite{Pfenning91lf}, hence no non-determinism arises
%in unification and no additional choices need to be stored in the proof
%witness\footnote{Susmit: please check this with Frank}. 

\section{Conclusion}
In this paper, we extended the logical framework LF with small proof
witnesses. Witness generation and checking  within the logical
framework LF constitutes a valuable addition to the general safety
infrastructure already provided, which not only supports
specification and execution of safety policies, but also can provide
insights into the relationship between safety policies and small
safety proofs and allows for experiments with different kinds of
encoding schemes.
 Given the potential of 
proof-carrying code methods and their new applications to
proof-carrying authorization \cite{AppelFelten99,bauer:thesis}, this
will provide a comprehensive guide for future implementations of proof
checkers which need not be restricted to first-order Prolog-like systems. 

Finally, small proof witnesses within the logical framework LF are
interesting independently of the proof-carrying code application. They
serve to increase the confidence in the logic programming and theorem
proving engines. As the size of the proof term can be
several orders of magnitude bigger than the query we are trying to
prove, constructing and possibly storing full proof terms during logic
programming proof search is expensive, and in some applications not even
feasible. Our experience and experiments with tabled higher-order
logic programming \cite{Pientka03phd} have demonstrated that
small proof  constructing and storing the full proof term 
however would  impose a substantial performance penalty. Therefore,
our implementation of tabled higher-order logic programming uses a
similar idea of small proof witnesses to construct proof witnesses
effectively without compromising the efficiency of the overall
system. 
% provide a cheap simple alternative and are in fact crucial to obtain a
% practical tabled logic programming interpreter. 

%Different notion of proof witnesses, called proof justifiers have been
%introduced in the logic programming community to ease debugging. Small proof
%witnesses can be viewed as proof justifiers and may be used to
%re-trace the proof.

\bibliographystyle{plain}
\bibliography{biblio}
\end{document}



