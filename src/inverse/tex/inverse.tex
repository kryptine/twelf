
\section{The Inverse Method}

\subsection{Requirements}

\begin{enumerate} 
\item Typechecker
  \begin{itemize} 
  \item Internal api, e.g. ``\%doublecheck''
  \item External api, for Deepak say.  Perhaps support explicit sharing of terms.
  \end{itemize} 
\item Term Reconstruction
\item Logic programming engine (Bottom up)
\item M2 Checker (\%mode,\%worlds,\%terminates,\%covers,\%total)
\item logic programming engine (Top down)
\item Meta-theorem prover (produce .prf files from .thm files)
\end{enumerate} 

\subsection{The Subformula Property}

\begin{Theorem} The cut rule is admissible in LF \end{Theorem} 
\XXX{Needs reference}

It is a property of cut-free sequent calculi that all propositions occurring
in a derivation are \emph{subformulas} of the endsequent in the following sense.

We define \emph{signed subformulas} of a formula $F$ with judgments:

\begin{align*} 
\PSubform{A} & & \mbox{$A$ is a positive subformula of the initial goal.}\\
\NSubform{A} & & \mbox{$A$ is a negative subformula of the initial goal.}
\end{align*} 

where 
$$\Psi ::= \circ \Spb \Psi, v:A \Spb \Psi, x:A$$

Here $v$ ranges over arbitrary terms and $x$ ranges over
fixed parameters.


These rules are defined as a forward logic program by the following inference
rules:

\bigskip 
\begin{tabular}{cc}
\infer{\NSubform{A}}{\PSubform{A\LFArrow B}} &
\infer{\NSubform{B}}{\PSubform{A\LFArrow B}} \\
\infer{\PSubform{A}}{\NSubform{A\LFArrow B}} &
\infer{\PSubform{B}}{\NSubform{A\LFArrow B}} \\
\infer{\PSubform[\Psi,x:A]{B}}{\PSubform{\PiTyp{x}{A}{B}}} &
\infer{\NSubform[\Psi,u:A]{B}}{\NSubform{\PiTyp{u}{A}{B}}} \\
\end{tabular} 
\bigskip 

Note that there are no rules for families of kind $\Type$.

To find the subformulas of a given goal formula, we run
the above rules as a forward logic program to saturation.  

\subsection{Admissible Substitutions}

Explicit substitutions are an integral part of the theory and 
implementation of our development.  They are necessary because
when logic variables arise during proof search, it is not always
possible to apply a substitution immediately.  While contexts 
do the necessary bookkeeping of tracking the types of bound variables,
substitutions track variable (term) instantiations.
The language of substitutions is

$$\sigma ::= \circ \Spb \sigma,(M/u) \Spb \sigma, (y//x)$$ 

where $M$ ranges over arbitrary terms, and $y$ ranges over
parameters.

An invariant(*) of admissible substitutions is that if $\AdmSub{\sigma}{\Psi}$ then
the domain of $\sigma$ is exactly the free variables of $\Psi$.  The codomain
need not be all of $\Sigma$.  Similarly, whenever we apply 
$A[\Theta]$ it will be the case that $\Theta$ substitutes exactly for the
free variables of $A$.  

The judgment on admissible substitutions will be 
$$\AdmSub{\sigma}{\Psi}$$
read ``$\sigma$ is an admissible substitution from $\Psi$ to $\Sigma$''.
and meaning that $\sigma$ substitutes for all the free variables in
$\Psi$, but can mention variables in $\Sigma$.  

\bigskip 

\infer{\AdmSub{\cdot}{\cdot}}{} 
\bigskip 

\infer{\AdmSub{(\sigma,M/u)}{\Psi,u:A}}{\AdmSub{\sigma}{\Psi} & \CheckTy[\Sigma]{M}{A[\sigma]}} 
\bigskip 

\infer{\AdmSub{(\sigma,y//x)}{\Psi,x:A}}{\AdmSub{\sigma}{\Psi} & \SynthTy[\Sigma]{y}{A'} & A' = \HerSub{A}} 
\bigskip 

\subsection{Forward Sequents}

For the inverse method, we think of the usual sequent calculus rules, which work bottom-up, 
backwards to work top-down.  

The judgment 

$$\Sequent{\Sigma}{A_1[\tau_1],\ldots,A_n[\tau_n]}{C[\sigma]}$$

where for all $i$, 

\begin{align*} 
  &\AdmSub{\tau_i}{\Psi_i}\\
  &\NSubform[\Psi_i]{A_i}\\
  &\AdmSub{\sigma}{\Psi}\\
  &\PSubform[\Psi]{C}
\end{align*} 

can be read as (modulo constraints on substitutions)
``Given with parameters $\Sigma$ and hypotheses and derived formulas 
$A_i$ under admissible substitutions $\tau_i$, the goal $C$ under admissible
substitution $\sigma$ is derivable.''

\subsection{Subsumption}

$$
\Sequent{\Sigma}{A_1[\tau_1],\ldots,A_n[\tau_n]}{C[\sigma]} \leq 
\Sequent{\Sigma'}{A_1[\tau_1'],\ldots,A_k[\tau_k']}{C[\sigma']}
$$

if there exists an admissible substitution $\AdmSub[Sigma']{\Theta}{\Sigma}$
such that 
$\tau_1' = \tau_1 \Comp \Theta, \ldots, \tau_k' = \tau_k \Comp \Theta, \sigma' = \sigma \Comp \Theta$

for some $n\leq k$.  

\subsection{Unification}

\subsection{Most General Unifiers}
In the pattern fragment of LF, most general unifiers exist.  Therefore the
function $\mgu(x,y)$ is well defined.  Keeping our invariant (*) in mind, 
we will write $\mgu = (\Theta_1,\Theta_2)$ when the domain of the intended
terms to which we apply the substitutions have different sets of free variables.
That is, if the most general unifier of $A_1,A_2$ is $\Theta$ and 
$fv(A_1) \neq fv(A_2)$ then 
\begin{align*} 
  \Theta_1 &= \mathbf{filter}\ (\lambda(x,\_).\ \mathbf{mem}(fv(A_1,x),x))\ \Theta \\ 
  \Theta_2 &= \mathbf{filter}\ (\lambda(x,\_).\ \mathbf{mem}(fv(A_2,x),x))\ \Theta
\end{align*} 


\subsection{Forward Sequent Rules, Pattern Fragment}

\infer[\textbf{Init}]{\Sequent{\Sigma}{P_1[\Theta_1]}{P_2[\Theta_2]}}{}
\bigskip 
for any $\NSubform[\Psi_1]{P_1}, \PSubform[\Psi_2]{P_2}$
such that $\MGU{(\Theta_1,\Theta_2)}{\NSubform[\Psi_1]{P_1}}{\PSubform[\Psi_2]{P_2}}$
with $\AdmSub{\Theta_1}{\Psi_1}$ and $\AdmSub{\Theta_2}{\Psi_2}$.

\bigskip 
\infer[\textbf{Contract}]{\Sequent{\Sigma'}{\Gamma[\Theta],A[\tau_1 \Comp \Theta}{C[\sigma \Comp \Theta}}
                         {\Sequent{\Sigma}{\Gamma,A[\tau_1],A[\tau_2]}{C[\sigma]}}

\bigskip 
when $\AdmSub{\tau_1}{\Psi}$, $\AdmSub{\tau_2}{\Psi}$, and $\MGU[\Sigma']{\Theta}{\tau_1}{\tau_2}$
\bigskip 

\infer[\textbf{ArrowR1}]{\Sequent{\Sigma'}{\Gamma[\Theta]}{(A\LFArrow B)[\tau \Comp \Theta]}}
                       {\Sequent{\Sigma}{\Gamma,A[\tau]}{B[\sigma]}}
\bigskip 
when $\MGU[\Sigma']{\Theta}{\tau}{\sigma}$, $\AdmSub{\Theta}{\Psi}$, $\AdmSub{\sigma}{\Psi}$, 
and $\PSubform{A \LFArrow B}$
\bigskip 

\infer[\textbf{ArrowR2}]{\Sequent{\Sigma}{\Gamma}{(A\LFArrow B)[\sigma]}}
                        {\Sequent{\Sigma}{\Gamma}{B[\sigma]}}
\bigskip 
when $\AdmSub{\sigma}{\Psi}$, and $\PSubform{A \LFArrow B}$
\bigskip 

\infer[\textbf{ArrowL}]{\Sequent{\Sigma}{\Gamma_1[\Theta_1],\Gamma_2[\Theta_2],(A\LFArrow B)[\tau_1 \Comp \Theta_1]}{C[\sigma \Comp \tau_2]}}
                       {\Sequent{\Sigma_1}{\Gamma_1}{A[\tau_1]} & \Sequent{\Sigma_2}{\Gamma_2,B[\tau_2]}{C[\sigma]}}
\bigskip 
when
\begin{align*} 
  &\MGU{(\Theta_1,\Theta_2)}{\AdmSub{\Theta_1}{\Sigma_1}}{\AdmSub{\Theta_2}{\Sigma_2}}\\
  &\AdmSub{\tau_1}{\Psi}\\
  &\AdmSub{\tau_2}{\Psi}\\
  &\NSubform{A \LFArrow B}
\end{align*} 
\bigskip 

\infer[\textbf{PiL}]{\Sequent{\Sigma}{\Gamma, (\PiTyp{u}{A}{B})[\tau]}{C[\sigma]}}
                    {\Sequent{\Sigma}{\Gamma, B[\tau,M/u]}{C[\sigma]}}
\bigskip 
when $\NSubform{\PiTyp{u}{A}{B}}$

\bigskip 
\infer[\textbf{PiR}]{\Sequent{\Sigma}{\Gamma}{(\PiTyp{x}{A}{B})[\sigma]}}
                    {\Sequent{\Sigma,y:A'}{\Gamma}{B[\sigma,y//x]}}
\bigskip 
when $y\not\in \Gamma$, $A' = A[\sigma]$ and $\PSubform{\PiTyp{x}{A}{B}}$

\subsection{Junk}


\newcommand{\init}{(\Gamma_0^-,C_0^+)}

\infer[$Atom$]{P \vdash P}{}
\bigskip 
when $P \leq^+ \init$ and $P \leq^- \init$
\bigskip 

\infer[$Contract$]{\Gamma,A \vdash B}{\Gamma,A,A\vdash B}
\bigskip 

\infer[\to$R1$]{\Gamma\vdash A \LFArrow B}{\Gamma,A \vdash B & (A\LFArrow B \leq^+ \init)}
\bigskip 

\infer[\LFArrow$R2$]{\Gamma\vdash A \LFArrow B}{\Gamma \vdash B & A \not\in\Gamma & (A\LFArrow B \leq^+ \init)}
\bigskip 

\infer[\LFArrow$L$]{\Gamma_1,\Gamma_2,A \LFArrow B\vdash C}{\Gamma_1,\vdash A,& \Gamma_2,A \vdash B & (A\LFArrow B \leq^- \init)}
\bigskip 

\subsection{The First Order Case}

\newcommand{\seq}[3][\Sigma]{#1 ; #2 \vdash #3}
\newcommand{\unif}[3][\Sigma]{#1 ; #2 = #3}
\newcommand{\renaming}[1]{#1\ \mathsf{renaming}}

General Form:

$$\Sigma; \Gamma[\tau] \vdash C[\sigma]$$

which is shorthand for

$$\Sigma; A_1[\tau_1],\ldots,A_n[\tau_n] \vdash C[\sigma]$$

\bigskip 

\infer[$Atom$]{\seq{P_1^-[\rho\circ\tau]}{P_2^+[\tau]}} {}
\bigskip 
when $\unif{P_1^-[\rho\circ\tau]}{P_2^+[\tau]}$ and $\renaming{\rho}$
\bigskip 

\infer[\Pi$-R$]{\seq{\Gamma}{}}{}
\bigskip 

